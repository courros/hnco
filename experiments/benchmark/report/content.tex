\graphicspath{{../graphics/}}

\section{Ranking}

\begin{center}
\begin{tabular}{@{}l*{10}{r}@{}}
\toprule
algorithm & \multicolumn{10}{l}{{rank distribution}}\\
\midrule
& 1 & 2 & 3 & 4 & 5 & 6 & 7 & 8 & 9 & 10\\
\midrule
sa & 11 & 6 & 1 & 1 & 0 & 0 & 0 & 0 & 0 & 0\\
umda & 8 & 6 & 1 & 2 & 2 & 0 & 0 & 0 & 0 & 0\\
ea-1c10 & 0 & 4 & 0 & 1 & 5 & 9 & 0 & 0 & 0 & 0\\
ea-1p1 & 0 & 2 & 9 & 3 & 1 & 2 & 2 & 0 & 0 & 0\\
pbil & 0 & 1 & 4 & 0 & 8 & 4 & 1 & 1 & 0 & 0\\
ea-1p10 & 0 & 0 & 4 & 10 & 2 & 3 & 0 & 0 & 0 & 0\\
rls & 0 & 0 & 0 & 2 & 0 & 0 & 3 & 8 & 5 & 1\\
ea-10p1 & 0 & 0 & 0 & 0 & 1 & 1 & 9 & 3 & 5 & 0\\
ga & 0 & 0 & 0 & 0 & 0 & 0 & 3 & 0 & 4 & 12\\
hc & 0 & 0 & 0 & 0 & 0 & 0 & 1 & 7 & 5 & 6\\
\bottomrule
\end{tabular}

\end{center}

Per function rankings (ex-eaquo are grouped in parentheses):
\begin{description}
\item[one-max]
umda, sa, ea-1p1, ea-1p10, ea-1c10, pbil, ga, ea-10p1, rls, hc

\item[lin]
umda, sa, ea-1p1, ea-1p10, ea-1c10, pbil, ea-10p1, rls, hc, ga

\item[leading-ones]
sa, umda, ea-1p1, ea-1p10, pbil, ea-1c10, ea-10p1, hc, rls, ga

\item[ridge]
sa, umda, ea-1p1, ea-1p10, ea-1c10, pbil, hc, ea-10p1, ga, rls

\item[jmp-5]
sa, ea-1c10, pbil, umda, ea-1p10, ea-1p1, rls, hc, ea-10p1, ga

\item[jmp-10]
sa, ea-1c10, pbil, rls, umda, ea-1p10, ea-1p1, hc, ea-10p1, ga

\item[djmp-5]
sa, ea-1c10, pbil, umda, ea-1p10, ea-1p1, rls, hc, ea-10p1, ga

\item[djmp-10]
sa, ea-1c10, pbil, rls, umda, ea-1p10, ea-1p1, hc, ea-10p1, ga

\item[fp-5]
sa, umda, ea-1p1, ea-1p10, pbil, ea-1c10, ea-10p1, hc, rls, ga

\item[fp-10]
sa, umda, ea-1p1, ea-1p10, pbil, ea-1c10, ea-10p1, hc, rls, ga

\item[nk]
umda, sa, ea-1p1, ea-1p10, pbil, ea-1c10, ea-10p1, rls, ga, hc

\item[max-sat]
umda, sa, ea-1p1, pbil, ea-1p10, ea-1c10, ea-10p1, rls, ga, hc

\item[labs]
sa, umda, ea-1p10, ea-1p1, ea-10p1, ea-1c10, pbil, rls, hc, ga

\item[ep]
umda, ea-1p1, ea-1p10, sa, ea-1c10, ea-10p1, rls, pbil, hc, ga

\item[cancel]
umda, sa, ea-1p1, ea-1p10, pbil, ea-1c10, ea-10p1, rls, hc, ga

\item[trap]
umda, sa, ea-1p1, ea-1p10, ea-1c10, pbil, ga, ea-10p1, rls, hc

\item[hiff]
sa, umda, ea-1p10, ea-1p1, pbil, ea-1c10, ea-10p1, rls, hc, ga

\item[plateau]
sa, pbil, umda, ea-1c10, ea-1p1, ea-1p10, ga, rls, ea-10p1, hc

\item[walsh2]
umda, sa, ea-1p1, ea-1p10, pbil, ea-1c10, ea-10p1, rls, ga, hc

\end{description}

\newpage

\section{Function one-max}

\begin{center}
\begin{tabular}{@{}l*{5}{>{{\nprounddigits{3}}}N{1}{3}}>{{\nprounddigits{0}}}N{2}{0}@{}}
\toprule
{algorithm} & \multicolumn{6}{l}{{cache lookup ratio}} \\
\midrule
& {min} & {$Q_1$} & {med.} & {$Q_3$} & {max} & {rk}\\
\midrule
rls & 0.135697 & 0.137920 & 0.139145 & 0.140553 & 0.144093 & 9\\
hc & 0.045840 & 0.046764 & 0.047100 & 0.047286 & 0.047933 & 10\\
sa & 0.895463 & 0.898539 & 0.901172 & 0.902012 & 0.904183 & 2\\
ea-1p1 & 0.863950 & 0.864531 & 0.864718 & 0.865035 & 0.866050 & 3\\
ea-1p10 & 0.862470 & 0.863355 & 0.863901 & 0.864511 & 0.864920 & 4\\
ea-10p1 & 0.143480 & 0.144673 & 0.145048 & 0.145818 & 0.147767 & 8\\
ea-1c10 & 0.848140 & 0.856130 & 0.858087 & 0.859271 & 0.867287 & 5\\
ga & 0.368067 & 0.371134 & 0.373117 & 0.375197 & 0.378833 & 7\\
pbil & 0.854090 & 0.854774 & 0.855152 & 0.855501 & 0.856103 & 6\\
umda & {\color{blue}} 0.903270 & {\color{blue}} 0.904007 & {\color{blue}} 0.904170 & {\color{blue}} 0.904560 & {\color{blue}} 0.905020 & 1\\
\bottomrule
\end{tabular}

\end{center}

\begin{center}
\begin{tabular}{@{}l*{6}{>{{\nprounddigits{2}}}N{1}{2}}@{}}
\toprule
{algorithm} & \multicolumn{2}{l}{{algo. time (s)}} & \multicolumn{2}{l}{{eval. time (s)}} & \multicolumn{2}{l}{{total time (s)}} \\
\midrule
& {mean} & {dev.} & {mean} & {dev.} & {mean} & {dev.} \\
\midrule
rls & 0.002111 & 0.001555 & 0.002230 & 0.001732 & 0.004342 & 0.003286 \\
 hc & 0.004135 & 0.000498 & 0.005098 & 0.000674 & 0.009233 & 0.001168 \\
 sa & 0.012959 & 0.001217 & 0.014660 & 0.001315 & 0.027618 & 0.002522 \\
 ea-1p1 & 0.000877 & 0.000245 & 0.000648 & 0.000206 & 0.001524 & 0.000449 \\
 ea-1p10 & 0.001162 & 0.000236 & 0.000844 & 0.000198 & 0.002007 & 0.000433 \\
 ea-10p1 & 0.010057 & 0.004031 & 0.007075 & 0.002893 & 0.017133 & 0.006921 \\
 ea-1c10 & 0.001684 & 0.000335 & 0.001384 & 0.000288 & 0.003068 & 0.000621 \\
 ga & 0.013693 & 0.003734 & 0.003246 & 0.000912 & 0.016939 & 0.004644 \\
 pbil & 0.078271 & 0.003628 & 0.015812 & 0.000787 & 0.094083 & 0.004390 \\
 umda & 0.005782 & 0.000808 & 0.001139 & 0.000177 & 0.006922 & 0.000982 \\
 \bottomrule
\end{tabular}

\end{center}

\begin{figure}[h]
\begin{center}
\includegraphics[width=0.6\linewidth]{one-max}
\caption{one-max}
\end{center}
\end{figure}

\begin{figure}[h]
\begin{center}
\includegraphics[width=0.6\linewidth]{one-max+all}
\caption{one-max}
\end{center}
\end{figure}

\newpage

\section{Function lin}

\begin{center}
\begin{tabular}{@{}l*{5}{>{{\nprounddigits{3}}}N{1}{3}}>{{\nprounddigits{0}}}N{2}{0}@{}}
\toprule
{algorithm} & \multicolumn{6}{l}{{cache lookup ratio}} \\
\midrule
& {min} & {$Q_1$} & {med.} & {$Q_3$} & {max} & {rk}\\
\midrule
rls & 0.133153 & 0.136928 & 0.138358 & 0.140589 & 0.145360 & 8\\
hc & 0.082560 & 0.084064 & 0.085665 & 0.090636 & 0.091733 & 9\\
sa & 0.863600 & 0.869990 & 0.872086 & 0.874188 & 0.884353 & 2\\
ea-1p1 & 0.862833 & 0.864313 & 0.864591 & 0.864875 & 0.865537 & 3\\
ea-1p10 & 0.862280 & 0.863248 & 0.863474 & 0.864006 & 0.864680 & 4\\
ea-10p1 & 0.732017 & 0.735572 & 0.736818 & 0.737651 & 0.742360 & 7\\
ea-1c10 & 0.848757 & 0.855367 & 0.856785 & 0.857963 & 0.867157 & 5\\
ga & 0.065367 & 0.067918 & 0.069662 & 0.071932 & 0.074797 & 10\\
pbil & 0.833220 & 0.836126 & 0.837077 & 0.837630 & 0.838873 & 6\\
umda & {\color{blue}} 0.901990 & {\color{blue}} 0.902455 & {\color{blue}} 0.902825 & {\color{blue}} 0.903178 & {\color{blue}} 0.903767 & 1\\
\bottomrule
\end{tabular}

\end{center}

\begin{center}
\begin{tabular}{@{}l*{6}{>{{\nprounddigits{2}}}N{1}{2}}@{}}
\toprule
{algorithm} & \multicolumn{2}{l}{{algo. time (s)}} & \multicolumn{2}{l}{{eval. time (s)}} & \multicolumn{2}{l}{{total time (s)}} \\
\midrule
& {mean} & {dev.} & {mean} & {dev.} & {mean} & {dev.} \\
\midrule
rls & 0.001734 & 0.001724 & 0.002734 & 0.002860 & 0.004468 & 0.004584 \\
 hc & 0.003087 & 0.000376 & 0.005910 & 0.000838 & 0.008996 & 0.001208 \\
 sa & 0.014218 & 0.002378 & 0.023505 & 0.003981 & 0.037723 & 0.006346 \\
 ea-1p1 & 0.000845 & 0.000203 & 0.000812 & 0.000211 & 0.001657 & 0.000412 \\
 ea-1p10 & 0.001183 & 0.000236 & 0.001105 & 0.000238 & 0.002288 & 0.000473 \\
 ea-10p1 & 0.008297 & 0.001742 & 0.006915 & 0.001449 & 0.015212 & 0.003185 \\
 ea-1c10 & 0.001434 & 0.000319 & 0.001553 & 0.000362 & 0.002987 & 0.000679 \\
 ga & 0.041043 & 0.025597 & 0.012837 & 0.007855 & 0.053880 & 0.033451 \\
 pbil & 0.086894 & 0.003480 & 0.026500 & 0.001021 & 0.113394 & 0.004484 \\
 umda & 0.007795 & 0.001048 & 0.002320 & 0.000328 & 0.010115 & 0.001372 \\
 \bottomrule
\end{tabular}

\end{center}

\begin{figure}[h]
\begin{center}
\includegraphics[width=0.6\linewidth]{lin}
\caption{lin}
\end{center}
\end{figure}

\begin{figure}[h]
\begin{center}
\includegraphics[width=0.6\linewidth]{lin+all}
\caption{lin}
\end{center}
\end{figure}

\newpage

\section{Function leading-ones}

\begin{center}
\begin{tabular}{@{}l*{5}{>{{\nprounddigits{3}}}N{1}{3}}>{{\nprounddigits{0}}}N{2}{0}@{}}
\toprule
{algorithm} & \multicolumn{6}{l}{{cache lookup ratio}} \\
\midrule
& {min} & {$Q_1$} & {med.} & {$Q_3$} & {max} & {rk}\\
\midrule
rls & 0.041717 & 0.042859 & 0.044682 & 0.045783 & 0.047307 & 9\\
hc & 0.080157 & 0.085526 & 0.087085 & 0.088277 & 0.091707 & 8\\
sa & 0.757713 & {\color{blue}} 0.980347 & {\color{blue}} 0.984192 & {\color{blue}} 0.986700 & {\color{blue}} 0.989713 & 1\\
ea-1p1 & 0.845833 & 0.849612 & 0.851023 & 0.853479 & 0.856107 & 3\\
ea-1p10 & 0.842140 & 0.847596 & 0.849400 & 0.852781 & 0.854043 & 4\\
ea-10p1 & 0.499377 & 0.505670 & 0.512841 & 0.522421 & 0.530983 & 7\\
ea-1c10 & 0.584063 & 0.604503 & 0.626463 & 0.631304 & 0.651160 & 6\\
ga & 0.007067 & 0.007302 & 0.007385 & 0.007475 & 0.008173 & 10\\
pbil & 0.649487 & 0.668238 & 0.676106 & 0.682031 & 0.690043 & 5\\
umda & {\color{blue}} 0.872080 & 0.874653 & 0.877193 & 0.879214 & 0.887290 & 2\\
\bottomrule
\end{tabular}

\end{center}

\begin{center}
\begin{tabular}{@{}l*{6}{>{{\nprounddigits{2}}}N{1}{2}}@{}}
\toprule
{algorithm} & \multicolumn{2}{l}{{algo. time (s)}} & \multicolumn{2}{l}{{eval. time (s)}} & \multicolumn{2}{l}{{total time (s)}} \\
\midrule
& {mean} & {dev.} & {mean} & {dev.} & {mean} & {dev.} \\
\midrule
rls & 0.025598 & 0.024169 & 0.030932 & 0.029304 & 0.056530 & 0.053472 \\
 hc & 0.004361 & 0.000564 & 0.005417 & 0.000742 & 0.009779 & 0.001300 \\
 sa & 0.007715 & 0.012827 & 0.008840 & 0.014911 & 0.016555 & 0.027738 \\
 ea-1p1 & 0.007079 & 0.001004 & 0.006147 & 0.000923 & 0.013226 & 0.001921 \\
 ea-1p10 & 0.007811 & 0.001279 & 0.006472 & 0.001117 & 0.014283 & 0.002394 \\
 ea-10p1 & 0.067446 & 0.010942 & 0.049795 & 0.008281 & 0.117241 & 0.019187 \\
 ea-1c10 & 0.013613 & 0.003077 & 0.012590 & 0.002801 & 0.026203 & 0.005875 \\
 ga & 1.593696 & 0.092971 & 0.389359 & 0.023511 & 1.983055 & 0.116024 \\
 pbil & 0.445765 & 0.021015 & 0.094694 & 0.004541 & 0.540460 & 0.025424 \\
 umda & 0.072713 & 0.013515 & 0.016088 & 0.003088 & 0.088801 & 0.016584 \\
 \bottomrule
\end{tabular}

\end{center}

\begin{figure}[h]
\begin{center}
\includegraphics[width=0.6\linewidth]{leading-ones}
\caption{leading-ones}
\end{center}
\end{figure}

\begin{figure}[h]
\begin{center}
\includegraphics[width=0.6\linewidth]{leading-ones+all}
\caption{leading-ones}
\end{center}
\end{figure}

\newpage

\section{Function ridge}

\begin{center}
\begin{tabular}{@{}l*{5}{>{{\nprounddigits{0}}}N{3}{0}}>{{\nprounddigits{0}}}N{2}{0}@{}}
\toprule
{algorithm} & \multicolumn{6}{l}{{function value}} \\
\midrule
& {min} & {$Q_1$} & {med.} & {$Q_3$} & {max} & {rk}\\
\midrule
rls & 103.000000 & 104.750000 & 105.500000 & 106.000000 & 108.000000 & 9\\
hc & {\color{blue}} 200.000000 & {\color{blue}} 200.000000 & {\color{blue}} 200.000000 & {\color{blue}} 200.000000 & {\color{blue}} 200.000000 & 1\\
sa & {\color{blue}} 200.000000 & {\color{blue}} 200.000000 & {\color{blue}} 200.000000 & {\color{blue}} 200.000000 & {\color{blue}} 200.000000 & 1\\
ea-1p1 & {\color{blue}} 200.000000 & {\color{blue}} 200.000000 & {\color{blue}} 200.000000 & {\color{blue}} 200.000000 & {\color{blue}} 200.000000 & 1\\
ea-1p10 & {\color{blue}} 200.000000 & {\color{blue}} 200.000000 & {\color{blue}} 200.000000 & {\color{blue}} 200.000000 & {\color{blue}} 200.000000 & 1\\
ea-10p1 & {\color{blue}} 200.000000 & {\color{blue}} 200.000000 & {\color{blue}} 200.000000 & {\color{blue}} 200.000000 & {\color{blue}} 200.000000 & 1\\
ea-1c10 & 119.000000 & 122.000000 & 124.500000 & 129.500000 & 138.000000 & 8\\
ga & 102.000000 & 103.000000 & 103.000000 & 103.000000 & 104.000000 & 10\\
pbil & 153.000000 & 154.000000 & 155.000000 & 156.000000 & 157.000000 & 7\\
umda & {\color{blue}} 200.000000 & {\color{blue}} 200.000000 & {\color{blue}} 200.000000 & {\color{blue}} 200.000000 & {\color{blue}} 200.000000 & 1\\
\bottomrule
\end{tabular}

\end{center}

\begin{center}
\begin{tabular}{@{}l*{6}{>{{\nprounddigits{2}}}N{1}{2}}@{}}
\toprule
{algorithm} & \multicolumn{2}{l}{{algo. time (s)}} & \multicolumn{2}{l}{{eval. time (s)}} & \multicolumn{2}{l}{{total time (s)}} \\
\midrule
& {mean} & {dev.} & {mean} & {dev.} & {mean} & {dev.} \\
\midrule
rls & 0.210290 & 0.004503 & 0.733048 & 0.008861 & 0.943338 & 0.010685 \\
 hc & 0.178121 & 0.002019 & 0.591234 & 0.010871 & 0.769355 & 0.012047 \\
 sa & 0.224151 & 0.006865 & 0.454839 & 0.003101 & 0.678989 & 0.006147 \\
 ea-1p1 & 0.315049 & 0.005557 & 0.530449 & 0.004083 & 0.845498 & 0.008719 \\
 ea-1p10 & 0.333315 & 0.002383 & 0.530107 & 0.003452 & 0.863423 & 0.005053 \\
 ea-10p1 & 0.403351 & 0.008574 & 0.630298 & 0.006825 & 1.033649 & 0.013941 \\
 ea-1c10 & 0.290204 & 0.001773 & 0.540797 & 0.003735 & 0.831001 & 0.004890 \\
 ga & 1.195045 & 0.003551 & 0.700952 & 0.008417 & 1.895998 & 0.009722 \\
 pbil & 1.261397 & 0.003078 & 0.548822 & 0.002608 & 1.810219 & 0.004904 \\
 umda & 1.239799 & 0.004118 & 0.519800 & 0.011372 & 1.759599 & 0.012825 \\
 \bottomrule
\end{tabular}

\end{center}

\begin{figure}[h]
\begin{center}
\includegraphics[width=0.6\linewidth]{ridge}
\caption{ridge}
\end{center}
\end{figure}

\begin{figure}[h]
\begin{center}
\includegraphics[width=0.6\linewidth]{ridge+all}
\caption{ridge}
\end{center}
\end{figure}

\newpage

\section{Function jmp-5}

\begin{center}
\begin{tabular}{@{}l*{5}{>{{\nprounddigits{3}}}N{1}{3}}>{{\nprounddigits{0}}}N{2}{0}@{}}
\toprule
{algorithm} & \multicolumn{6}{l}{{cache lookup ratio}} \\
\midrule
& {min} & {$Q_1$} & {med.} & {$Q_3$} & {max} & {rk}\\
\midrule
rls & 0.087993 & 0.088957 & 0.089190 & 0.089653 & 0.089923 & 7\\
hc & 0.019323 & 0.019332 & 0.019337 & 0.019343 & 0.019357 & 8\\
sa & {\color{blue}} 0.784707 & {\color{blue}} 0.787156 & {\color{blue}} 0.789625 & {\color{blue}} 0.791056 & {\color{blue}} 0.793653 & 1\\
ea-1p1 & 0.099993 & 0.102543 & 0.102917 & 0.103953 & 0.105170 & 6\\
ea-1p10 & 0.101210 & 0.102656 & 0.103300 & 0.104228 & 0.107493 & 5\\
ea-10p1 & 0.012073 & 0.012622 & 0.012730 & 0.012890 & 0.013223 & 9\\
ea-1c10 & 0.507090 & 0.512478 & 0.513732 & 0.516478 & 0.520057 & 2\\
ga & 0.004113 & 0.004172 & 0.004267 & 0.004300 & 0.004357 & 10\\
pbil & 0.188763 & 0.318110 & 0.355347 & 0.400178 & 0.536200 & 3\\
umda & 0.159153 & 0.173653 & 0.181059 & 0.187833 & 0.209957 & 4\\
\bottomrule
\end{tabular}

\end{center}

\begin{center}
\begin{tabular}{@{}l*{6}{>{{\nprounddigits{2}}}N{1}{2}}@{}}
\toprule
{algorithm} & \multicolumn{2}{l}{{algo. time (s)}} & \multicolumn{2}{l}{{eval. time (s)}} & \multicolumn{2}{l}{{total time (s)}} \\
\midrule
& {mean} & {dev.} & {mean} & {dev.} & {mean} & {dev.} \\
\midrule
rls & 0.217435 & 0.001726 & 0.277032 & 0.002179 & 0.494467 & 0.003019 \\
 hc & 0.186589 & 0.001818 & 0.277223 & 0.003350 & 0.463812 & 0.004254 \\
 sa & 0.223558 & 0.004699 & 0.277940 & 0.003001 & 0.501498 & 0.006584 \\
 ea-1p1 & 0.320865 & 0.004767 & 0.279246 & 0.005983 & 0.600111 & 0.008009 \\
 ea-1p10 & 0.342886 & 0.002130 & 0.279568 & 0.002908 & 0.622454 & 0.004432 \\
 ea-10p1 & 0.445595 & 0.009544 & 0.281141 & 0.004860 & 0.726736 & 0.012026 \\
 ea-1c10 & 0.299998 & 0.002574 & 0.278169 & 0.003253 & 0.578167 & 0.004834 \\
 ga & 0.463696 & 0.448818 & 0.104132 & 0.100384 & 0.567828 & 0.549197 \\
 pbil & 0.069969 & 0.003395 & 0.013699 & 0.000706 & 0.083668 & 0.004090 \\
 umda & 0.184370 & 0.190956 & 0.040433 & 0.042290 & 0.224804 & 0.233241 \\
 \bottomrule
\end{tabular}

\end{center}

\begin{figure}[h]
\begin{center}
\includegraphics[width=0.6\linewidth]{jmp-5}
\caption{jmp-5}
\end{center}
\end{figure}

\begin{figure}[h]
\begin{center}
\includegraphics[width=0.6\linewidth]{jmp-5+all}
\caption{jmp-5}
\end{center}
\end{figure}

\newpage

\section{Function jmp-10}

\begin{center}
\begin{tabular}{@{}l*{5}{>{{\nprounddigits{3}}}N{1}{3}}>{{\nprounddigits{0}}}N{2}{0}@{}}
\toprule
{algorithm} & \multicolumn{6}{l}{{cache lookup ratio}} \\
\midrule
& {min} & {$Q_1$} & {med.} & {$Q_3$} & {max} & {rk}\\
\midrule
rls & 0.085463 & 0.085933 & 0.086255 & 0.086663 & 0.087037 & 4\\
hc & 0.019243 & 0.019253 & 0.019257 & 0.019267 & 0.019277 & 8\\
sa & {\color{blue}} 0.780243 & {\color{blue}} 0.785615 & {\color{blue}} 0.788037 & {\color{blue}} 0.792188 & {\color{blue}} 0.795427 & 1\\
ea-1p1 & 0.055890 & 0.057442 & 0.058222 & 0.058638 & 0.060007 & 7\\
ea-1p10 & 0.058470 & 0.059236 & 0.059348 & 0.059912 & 0.060760 & 6\\
ea-10p1 & 0.009627 & 0.009766 & 0.009863 & 0.010056 & 0.010297 & 9\\
ea-1c10 & 0.464890 & 0.469908 & 0.471209 & 0.472389 & 0.474583 & 2\\
ga & 0.003327 & 0.003452 & 0.003480 & 0.003552 & 0.003747 & 10\\
pbil & 0.118573 & 0.205374 & 0.263707 & 0.319388 & 0.373117 & 3\\
umda & 0.061333 & 0.067624 & 0.073060 & 0.076623 & 0.083083 & 5\\
\bottomrule
\end{tabular}

\end{center}

\begin{center}
\begin{tabular}{@{}l*{6}{>{{\nprounddigits{2}}}N{1}{2}}@{}}
\toprule
{algorithm} & \multicolumn{2}{l}{{algo. time (s)}} & \multicolumn{2}{l}{{eval. time (s)}} & \multicolumn{2}{l}{{total time (s)}} \\
\midrule
& {mean} & {dev.} & {mean} & {dev.} & {mean} & {dev.} \\
\midrule
rls & 0.367082 & 0.004592 & 0.423358 & 0.006648 & 0.790440 & 0.009916 \\
 hc & 0.326486 & 0.005151 & 0.423178 & 0.003785 & 0.749663 & 0.004333 \\
 sa & 0.361002 & 0.002755 & 0.428345 & 0.006165 & 0.789346 & 0.005819 \\
 ea-1p1 & 0.467289 & 0.005691 & 0.417945 & 0.008637 & 0.885234 & 0.013315 \\
 ea-1p10 & 0.489894 & 0.003466 & 0.414323 & 0.006134 & 0.904217 & 0.006137 \\
 ea-10p1 & 0.571430 & 0.011564 & 0.421123 & 0.009866 & 0.992552 & 0.018959 \\
 ea-1c10 & 0.448878 & 0.006450 & 0.418598 & 0.008716 & 0.867476 & 0.012216 \\
 ga & 1.576601 & 0.006396 & 0.392400 & 0.006481 & 1.969001 & 0.008058 \\
 pbil & 1.044520 & 0.760895 & 0.246747 & 0.180908 & 1.291267 & 0.941752 \\
 umda & 1.611554 & 0.014277 & 0.390084 & 0.006543 & 2.001639 & 0.018480 \\
 \bottomrule
\end{tabular}

\end{center}

\begin{figure}[h]
\begin{center}
\includegraphics[width=0.6\linewidth]{jmp-10}
\caption{jmp-10}
\end{center}
\end{figure}

\begin{figure}[h]
\begin{center}
\includegraphics[width=0.6\linewidth]{jmp-10+all}
\caption{jmp-10}
\end{center}
\end{figure}

\newpage

\section{Function djmp-5}

\begin{center}
\begin{tabular}{@{}l*{5}{>{{\nprounddigits{3}}}N{1}{3}}>{{\nprounddigits{0}}}N{2}{0}@{}}
\toprule
{algorithm} & \multicolumn{6}{l}{{cache lookup ratio}} \\
\midrule
& {min} & {$Q_1$} & {med.} & {$Q_3$} & {max} & {rk}\\
\midrule
rls & 0.088053 & 0.088818 & 0.089283 & 0.089706 & 0.089963 & 7\\
hc & 0.019313 & 0.019333 & 0.019337 & 0.019344 & 0.019357 & 8\\
sa & {\color{blue}} 0.783703 & {\color{blue}} 0.786064 & {\color{blue}} 0.787760 & {\color{blue}} 0.788730 & {\color{blue}} 0.793030 & 1\\
ea-1p1 & 0.100563 & 0.101595 & 0.102857 & 0.103807 & 0.105583 & 6\\
ea-1p10 & 0.100893 & 0.102255 & 0.103998 & 0.104635 & 0.105207 & 5\\
ea-10p1 & 0.012180 & 0.012627 & 0.012752 & 0.012946 & 0.013243 & 9\\
ea-1c10 & 0.509207 & 0.513908 & 0.515197 & 0.516439 & 0.518307 & 2\\
ga & 0.004097 & 0.004244 & 0.004297 & 0.004357 & 0.004497 & 10\\
pbil & 0.248647 & 0.284116 & 0.343001 & 0.387569 & 0.446740 & 3\\
umda & 0.164553 & 0.175880 & 0.180852 & 0.185209 & 0.202587 & 4\\
\bottomrule
\end{tabular}

\end{center}

\begin{center}
\begin{tabular}{@{}l*{6}{>{{\nprounddigits{2}}}N{1}{2}}@{}}
\toprule
{algorithm} & \multicolumn{2}{l}{{algo. time (s)}} & \multicolumn{2}{l}{{eval. time (s)}} & \multicolumn{2}{l}{{total time (s)}} \\
\midrule
& {mean} & {dev.} & {mean} & {dev.} & {mean} & {dev.} \\
\midrule
rls & 0.362827 & 0.009335 & 0.424194 & 0.012895 & 0.787021 & 0.020971 \\
 hc & 0.325596 & 0.003151 & 0.422217 & 0.005583 & 0.747813 & 0.004411 \\
 sa & 0.360390 & 0.003208 & 0.429274 & 0.004229 & 0.789664 & 0.002633 \\
 ea-1p1 & 0.466548 & 0.006551 & 0.417669 & 0.010120 & 0.884216 & 0.014361 \\
 ea-1p10 & 0.490137 & 0.004175 & 0.417092 & 0.007176 & 0.907228 & 0.007923 \\
 ea-10p1 & 0.570318 & 0.009442 & 0.422287 & 0.011547 & 0.992605 & 0.019202 \\
 ea-1c10 & 0.448141 & 0.006609 & 0.414661 & 0.005697 & 0.862803 & 0.009204 \\
 ga & 0.473677 & 0.414898 & 0.117714 & 0.102296 & 0.591391 & 0.517181 \\
 pbil & 0.088998 & 0.004501 & 0.019782 & 0.001093 & 0.108780 & 0.005577 \\
 umda & 0.186011 & 0.225217 & 0.045578 & 0.055600 & 0.231590 & 0.280808 \\
 \bottomrule
\end{tabular}

\end{center}

\begin{figure}[h]
\begin{center}
\includegraphics[width=0.6\linewidth]{djmp-5}
\caption{djmp-5}
\end{center}
\end{figure}

\begin{figure}[h]
\begin{center}
\includegraphics[width=0.6\linewidth]{djmp-5+all}
\caption{djmp-5}
\end{center}
\end{figure}

\newpage

\section{Function djmp-10}

\begin{center}
\begin{tabular}{@{}l*{5}{>{{\nprounddigits{3}}}N{1}{3}}>{{\nprounddigits{0}}}N{2}{0}@{}}
\toprule
{algorithm} & \multicolumn{6}{l}{{cache lookup ratio}} \\
\midrule
& {min} & {$Q_1$} & {med.} & {$Q_3$} & {max} & {rk}\\
\midrule
rls & 0.085327 & 0.085751 & 0.086148 & 0.086453 & 0.087380 & 4\\
hc & 0.019240 & 0.019253 & 0.019260 & 0.019267 & 0.019280 & 8\\
sa & {\color{blue}} 0.783443 & {\color{blue}} 0.786571 & {\color{blue}} 0.788842 & {\color{blue}} 0.790839 & {\color{blue}} 0.796430 & 1\\
ea-1p1 & 0.057300 & 0.057838 & 0.058275 & 0.058582 & 0.059457 & 7\\
ea-1p10 & 0.057727 & 0.058855 & 0.059225 & 0.059894 & 0.060820 & 6\\
ea-10p1 & 0.009467 & 0.009753 & 0.009887 & 0.010031 & 0.010173 & 9\\
ea-1c10 & 0.466323 & 0.467772 & 0.469660 & 0.471332 & 0.474970 & 2\\
ga & 0.003387 & 0.003439 & 0.003508 & 0.003573 & 0.003697 & 10\\
pbil & 0.162237 & 0.267775 & 0.305775 & 0.333804 & 0.415063 & 3\\
umda & 0.061373 & 0.071821 & 0.076497 & 0.078352 & 0.081040 & 5\\
\bottomrule
\end{tabular}

\end{center}

\begin{center}
\begin{tabular}{@{}l*{6}{>{{\nprounddigits{2}}}N{1}{2}}@{}}
\toprule
{algorithm} & \multicolumn{2}{l}{{algo. time (s)}} & \multicolumn{2}{l}{{eval. time (s)}} & \multicolumn{2}{l}{{total time (s)}} \\
\midrule
& {mean} & {dev.} & {mean} & {dev.} & {mean} & {dev.} \\
\midrule
rls & 0.217539 & 0.004975 & 0.739097 & 0.007240 & 0.956636 & 0.010382 \\
 hc & 0.180757 & 0.001719 & 0.744934 & 0.009509 & 0.925691 & 0.009875 \\
 sa & 0.215553 & 0.001692 & 0.502533 & 0.003234 & 0.718085 & 0.004504 \\
 ea-1p1 & 0.325622 & 0.009033 & 0.760559 & 0.011390 & 1.086180 & 0.014321 \\
 ea-1p10 & 0.340376 & 0.003778 & 0.757662 & 0.006713 & 1.098037 & 0.008193 \\
 ea-10p1 & 0.420826 & 0.013196 & 0.789051 & 0.008128 & 1.209877 & 0.019157 \\
 ea-1c10 & 0.292636 & 0.002222 & 0.601642 & 0.004926 & 0.894279 & 0.006397 \\
 ga & 1.197111 & 0.003054 & 0.804751 & 0.005902 & 2.001862 & 0.006732 \\
 pbil & 1.326769 & 0.008131 & 0.718259 & 0.035279 & 2.045027 & 0.041836 \\
 umda & 1.295626 & 0.003851 & 0.761358 & 0.008319 & 2.056984 & 0.007453 \\
 \bottomrule
\end{tabular}

\end{center}

\begin{figure}[h]
\begin{center}
\includegraphics[width=0.6\linewidth]{djmp-10}
\caption{djmp-10}
\end{center}
\end{figure}

\begin{figure}[h]
\begin{center}
\includegraphics[width=0.6\linewidth]{djmp-10+all}
\caption{djmp-10}
\end{center}
\end{figure}

\newpage

\section{Function fp-5}

\begin{center}
\begin{tabular}{@{}l*{5}{>{{\nprounddigits{3}}}N{1}{3}}>{{\nprounddigits{0}}}N{2}{0}@{}}
\toprule
{algorithm} & \multicolumn{6}{l}{{cache lookup ratio}} \\
\midrule
& {min} & {$Q_1$} & {med.} & {$Q_3$} & {max} & {rk}\\
\midrule
rls & 0.039873 & 0.040526 & 0.042295 & 0.042815 & 0.046817 & 9\\
hc & 0.064003 & 0.068854 & 0.071592 & 0.072873 & 0.074590 & 8\\
sa & 0.588360 & {\color{blue}} 0.982139 & {\color{blue}} 0.984174 & {\color{blue}} 0.985593 & {\color{blue}} 0.990330 & 1\\
ea-1p1 & 0.846000 & 0.850165 & 0.851946 & 0.852715 & 0.856290 & 3\\
ea-1p10 & 0.845880 & 0.849836 & 0.851762 & 0.852625 & 0.855117 & 4\\
ea-10p1 & 0.486497 & 0.495220 & 0.516339 & 0.521228 & 0.539353 & 7\\
ea-1c10 & 0.594230 & 0.612667 & 0.623450 & 0.635428 & 0.653517 & 6\\
ga & 0.007187 & 0.007343 & 0.007577 & 0.007775 & 0.007993 & 10\\
pbil & 0.646360 & 0.658014 & 0.667830 & 0.673388 & 0.688497 & 5\\
umda & {\color{blue}} 0.868230 & 0.873588 & 0.877800 & 0.881060 & 0.885373 & 2\\
\bottomrule
\end{tabular}

\end{center}

\begin{center}
\begin{tabular}{@{}l*{6}{>{{\nprounddigits{2}}}N{1}{2}}@{}}
\toprule
{algorithm} & \multicolumn{2}{l}{{algo. time (s)}} & \multicolumn{2}{l}{{eval. time (s)}} & \multicolumn{2}{l}{{total time (s)}} \\
\midrule
& {mean} & {dev.} & {mean} & {dev.} & {mean} & {dev.} \\
\midrule
rls & 0.011591 & 0.014031 & 0.015804 & 0.019302 & 0.027394 & 0.033332 \\
 hc & 0.152674 & 0.054967 & 0.229311 & 0.082860 & 0.381984 & 0.137806 \\
 sa & 0.014656 & 0.043035 & 0.019007 & 0.055969 & 0.033662 & 0.099004 \\
 ea-1p1 & 0.005840 & 0.001116 & 0.004913 & 0.000941 & 0.010753 & 0.002052 \\
 ea-1p10 & 0.006214 & 0.001347 & 0.004924 & 0.001085 & 0.011137 & 0.002430 \\
 ea-10p1 & 0.057457 & 0.008488 & 0.036712 & 0.005413 & 0.094169 & 0.013878 \\
 ea-1c10 & 0.013030 & 0.009131 & 0.012156 & 0.008722 & 0.025186 & 0.017851 \\
 ga & 1.236970 & 0.063730 & 0.285732 & 0.014728 & 1.522702 & 0.078171 \\
 pbil & 0.355650 & 0.034567 & 0.076186 & 0.008071 & 0.431836 & 0.042575 \\
 umda & 0.049147 & 0.005841 & 0.010931 & 0.001378 & 0.060078 & 0.007213 \\
 \bottomrule
\end{tabular}

\end{center}

\begin{figure}[h]
\begin{center}
\includegraphics[width=0.6\linewidth]{fp-5}
\caption{fp-5}
\end{center}
\end{figure}

\begin{figure}[h]
\begin{center}
\includegraphics[width=0.6\linewidth]{fp-5+all}
\caption{fp-5}
\end{center}
\end{figure}

\newpage

\section{Function fp-10}

\begin{center}
\begin{tabular}{@{}l*{5}{>{{\nprounddigits{3}}}N{1}{3}}>{{\nprounddigits{0}}}N{2}{0}@{}}
\toprule
{algorithm} & \multicolumn{6}{l}{{cache lookup ratio}} \\
\midrule
& {min} & {$Q_1$} & {med.} & {$Q_3$} & {max} & {rk}\\
\midrule
rls & 0.036073 & 0.037013 & 0.038608 & 0.039302 & 0.041103 & 9\\
hc & 0.065957 & 0.069100 & 0.070598 & 0.072177 & 0.075833 & 8\\
sa & 0.808053 & {\color{blue}} 0.980751 & {\color{blue}} 0.984225 & {\color{blue}} 0.986277 & {\color{blue}} 0.991287 & 1\\
ea-1p1 & 0.846460 & 0.850420 & 0.851775 & 0.853177 & 0.859123 & 3\\
ea-1p10 & 0.842130 & 0.847677 & 0.850212 & 0.852905 & 0.857200 & 4\\
ea-10p1 & 0.467657 & 0.502751 & 0.512764 & 0.520281 & 0.533847 & 7\\
ea-1c10 & 0.574113 & 0.602752 & 0.611552 & 0.633168 & 0.664287 & 6\\
ga & 0.006913 & 0.007296 & 0.007398 & 0.007547 & 0.008030 & 10\\
pbil & 0.657663 & 0.671189 & 0.678885 & 0.688848 & 0.702137 & 5\\
umda & {\color{blue}} 0.872893 & 0.877128 & 0.880167 & 0.882659 & 0.888870 & 2\\
\bottomrule
\end{tabular}

\end{center}

\begin{center}
\begin{tabular}{@{}l*{6}{>{{\nprounddigits{2}}}N{1}{2}}@{}}
\toprule
{algorithm} & \multicolumn{2}{l}{{algo. time (s)}} & \multicolumn{2}{l}{{eval. time (s)}} & \multicolumn{2}{l}{{total time (s)}} \\
\midrule
& {mean} & {dev.} & {mean} & {dev.} & {mean} & {dev.} \\
\midrule
rls & 0.201191 & 0.003332 & 0.766093 & 0.010482 & 0.967284 & 0.011805 \\
 hc & 0.181459 & 0.002480 & 0.747885 & 0.004867 & 0.929344 & 0.006245 \\
 sa & 0.209567 & 0.003503 & 0.454005 & 0.033945 & 0.663572 & 0.032815 \\
 ea-1p1 & 0.316716 & 0.006343 & 0.526222 & 0.006610 & 0.842938 & 0.011092 \\
 ea-1p10 & 0.334398 & 0.002490 & 0.521533 & 0.003550 & 0.855931 & 0.004725 \\
 ea-10p1 & 0.403132 & 0.009594 & 0.662460 & 0.010961 & 1.065592 & 0.016322 \\
 ea-1c10 & 0.291440 & 0.002151 & 0.585193 & 0.015401 & 0.876634 & 0.015970 \\
 ga & 1.205738 & 0.003862 & 0.840661 & 0.017827 & 2.046399 & 0.017639 \\
 pbil & 1.275211 & 0.004346 & 0.572744 & 0.008919 & 1.847954 & 0.010494 \\
 umda & 1.244471 & 0.004454 & 0.516620 & 0.014116 & 1.761091 & 0.014514 \\
 \bottomrule
\end{tabular}

\end{center}

\begin{figure}[h]
\begin{center}
\includegraphics[width=0.6\linewidth]{fp-10}
\caption{fp-10}
\end{center}
\end{figure}

\begin{figure}[h]
\begin{center}
\includegraphics[width=0.6\linewidth]{fp-10+all}
\caption{fp-10}
\end{center}
\end{figure}

\newpage

\section{Function nk}

\begin{center}
\begin{tabular}{@{}l*{5}{>{{\nprounddigits{2}}}N{1}{2}}>{{\nprounddigits{0}}}N{2}{0}@{}}
\toprule
{algorithm} & \multicolumn{6}{l}{{function value}} \\
\midrule
& {min} & {$Q_1$} & {med.} & {$Q_3$} & {max} & {rk}\\
\midrule
rls & 0.960631 & 0.975590 & 0.992241 & 1.005653 & 1.031750 & 5\\
hc & 0.958381 & 0.984648 & 0.997946 & 1.012398 & 1.039900 & 4\\
sa & {\color{blue}} 1.016100 & {\color{blue}} 1.045970 & {\color{blue}} 1.062635 & {\color{blue}} 1.075625 & {\color{blue}} 1.102760 & 1\\
ea-1p1 & 0.820294 & 0.903381 & 0.949750 & 0.980328 & 1.014600 & 8\\
ea-1p10 & 0.838381 & 0.895871 & 0.931623 & 0.982986 & 1.043010 & 9\\
ea-10p1 & 0.844339 & 0.948574 & 0.985073 & 1.003900 & {\color{blue}} 1.102760 & 6\\
ea-1c10 & 0.939837 & 1.007710 & 1.036105 & 1.055157 & 1.085990 & 3\\
ga & 0.980635 & 1.017737 & 1.038020 & 1.061627 & 1.072170 & 2\\
pbil & 0.950140 & 0.967605 & 0.981174 & 0.999697 & 1.023370 & 7\\
umda & 0.799176 & 0.898308 & 0.925997 & 0.969150 & 1.021880 & 10\\
\bottomrule
\end{tabular}

\end{center}

\begin{center}
\begin{tabular}{@{}l*{6}{>{{\nprounddigits{2}}}N{1}{2}}@{}}
\toprule
{algorithm} & \multicolumn{2}{l}{{algo. time (s)}} & \multicolumn{2}{l}{{eval. time (s)}} & \multicolumn{2}{l}{{total time (s)}} \\
\midrule
& {mean} & {dev.} & {mean} & {dev.} & {mean} & {dev.} \\
\midrule
rls & 0.361770 & 0.009802 & 1.096436 & 0.023789 & 1.458206 & 0.032494 \\
 hc & 0.321910 & 0.008494 & 1.045990 & 0.023966 & 1.367900 & 0.031705 \\
 sa & 0.371192 & 0.009521 & 1.022185 & 0.026385 & 1.393377 & 0.034497 \\
 ea-1p1 & 0.477042 & 0.006937 & 1.076246 & 0.018798 & 1.553287 & 0.023280 \\
 ea-1p10 & 0.493790 & 0.008060 & 1.061377 & 0.015098 & 1.555167 & 0.020864 \\
 ea-10p1 & 0.586578 & 0.017318 & 1.107001 & 0.016531 & 1.693579 & 0.029689 \\
 ea-1c10 & 0.452337 & 0.010567 & 1.001919 & 0.025011 & 1.454256 & 0.034465 \\
 ga & 1.718506 & 0.031091 & 1.207643 & 0.024718 & 2.926150 & 0.052494 \\
 pbil & 1.865412 & 0.024163 & 1.059262 & 0.017250 & 2.924674 & 0.038548 \\
 umda & 1.835314 & 0.020308 & 0.979173 & 0.020181 & 2.814487 & 0.034670 \\
 \bottomrule
\end{tabular}

\end{center}

\begin{figure}[h]
\begin{center}
\includegraphics[width=0.6\linewidth]{nk}
\caption{nk}
\end{center}
\end{figure}

\begin{figure}[h]
\begin{center}
\includegraphics[width=0.6\linewidth]{nk+all}
\caption{nk}
\end{center}
\end{figure}

\newpage

\section{Function max-sat}

\begin{center}
\begin{tabular}{@{}l*{5}{>{{\nprounddigits{0}}}N{3}{0}}>{{\nprounddigits{0}}}N{2}{0}@{}}
\toprule
{algorithm} & \multicolumn{6}{l}{{function value}} \\
\midrule
& {min} & {$Q_1$} & {med.} & {$Q_3$} & {max} & {rk}\\
\midrule
rls & {\color{blue}} 971.000000 & {\color{blue}} 972.000000 & {\color{blue}} 972.000000 & {\color{blue}} 972.000000 & {\color{blue}} 972.000000 & 1\\
hc & 963.000000 & 965.000000 & 966.000000 & 967.250000 & 971.000000 & 9\\
sa & {\color{blue}} 971.000000 & {\color{blue}} 972.000000 & {\color{blue}} 972.000000 & {\color{blue}} 972.000000 & {\color{blue}} 972.000000 & 1\\
ea-1p1 & 959.000000 & 964.000000 & 966.000000 & 968.250000 & {\color{blue}} 972.000000 & 10\\
ea-1p10 & 958.000000 & 962.750000 & 967.000000 & 968.000000 & 971.000000 & 8\\
ea-10p1 & 962.000000 & 966.500000 & 968.000000 & 969.000000 & {\color{blue}} 972.000000 & 5\\
ea-1c10 & 964.000000 & 968.000000 & 969.000000 & 971.000000 & {\color{blue}} 972.000000 & 3\\
ga & 965.000000 & 967.750000 & 968.000000 & {\color{blue}} 972.000000 & {\color{blue}} 972.000000 & 4\\
pbil & 964.000000 & 966.500000 & 967.000000 & 967.000000 & 968.000000 & 7\\
umda & 963.000000 & 964.000000 & 967.500000 & 971.000000 & {\color{blue}} 972.000000 & 6\\
\bottomrule
\end{tabular}

\end{center}

\begin{center}
\begin{tabular}{@{}l*{6}{>{{\nprounddigits{2}}}N{1}{2}}@{}}
\toprule
{algorithm} & \multicolumn{2}{l}{{algo. time (s)}} & \multicolumn{2}{l}{{eval. time (s)}} & \multicolumn{2}{l}{{total time (s)}} \\
\midrule
& {mean} & {dev.} & {mean} & {dev.} & {mean} & {dev.} \\
\midrule
rls & 0.319148 & 0.002486 & 4.456952 & 0.127115 & 4.776100 & 0.126869 \\
 hc & 0.288248 & 0.004281 & 4.030953 & 0.089631 & 4.319200 & 0.091116 \\
 sa & 0.338689 & 0.009020 & 3.733824 & 0.109818 & 4.072513 & 0.104033 \\
 ea-1p1 & 0.456728 & 0.009532 & 4.061210 & 0.117852 & 4.517938 & 0.115367 \\
 ea-1p10 & 0.457719 & 0.004729 & 4.039699 & 0.119193 & 4.497417 & 0.120398 \\
 ea-10p1 & 0.535625 & 0.011475 & 4.923000 & 0.077301 & 5.458625 & 0.079804 \\
 ea-1c10 & 0.416184 & 0.002388 & 3.618575 & 0.058791 & 4.034759 & 0.059201 \\
 ga & 1.526335 & 0.003055 & 5.264863 & 0.095471 & 6.791198 & 0.094732 \\
 pbil & 1.585288 & 0.009487 & 4.077560 & 0.122032 & 5.662848 & 0.127891 \\
 umda & 1.519071 & 0.004014 & 3.868872 & 0.059580 & 5.387944 & 0.059520 \\
 \bottomrule
\end{tabular}

\end{center}

\begin{figure}[h]
\begin{center}
\includegraphics[width=0.6\linewidth]{max-sat}
\caption{max-sat}
\end{center}
\end{figure}

\begin{figure}[h]
\begin{center}
\includegraphics[width=0.6\linewidth]{max-sat+all}
\caption{max-sat}
\end{center}
\end{figure}

\newpage

\section{Function labs}

\begin{center}
\begin{tabular}{@{}l*{5}{>{{\nprounddigits{2}}}N{1}{2}}>{{\nprounddigits{0}}}N{2}{0}@{}}
\toprule
{algorithm} & \multicolumn{6}{l}{{function value}} \\
\midrule
& {min} & {$Q_1$} & {med.} & {$Q_3$} & {max} & {rk}\\
\midrule
rls & 4.244480 & 4.336605 & 4.409225 & 4.500550 & 4.835590 & 6\\
hc & 4.587160 & 4.721450 & 4.771005 & 4.859100 & 5.010020 & 3\\
sa & 4.393670 & 4.621260 & 4.699265 & 4.935853 & {\color{blue}} 5.446620 & 4\\
ea-1p1 & 3.671070 & 3.826093 & 3.962125 & 4.029013 & 5.010020 & 8\\
ea-1p10 & 3.692760 & 3.864098 & 4.105265 & 4.413073 & 4.816960 & 7\\
ea-10p1 & 4.159730 & 4.537380 & 4.655490 & 4.821938 & 5.376340 & 5\\
ea-1c10 & 4.570380 & 4.793873 & 4.863830 & 5.071250 & 5.285410 & 2\\
ga & {\color{blue}} 4.690430 & {\color{blue}} 4.849675 & {\color{blue}} 4.960495 & {\color{blue}} 5.197510 & 5.353320 & 1\\
pbil & 3.477050 & 3.615337 & 3.863990 & 4.055805 & 4.230120 & 10\\
umda & 3.217500 & 3.652353 & 3.900310 & 4.065040 & 4.288160 & 9\\
\bottomrule
\end{tabular}

\end{center}

\begin{center}
\begin{tabular}{@{}l*{6}{>{{\nprounddigits{2}}}N{1}{2}}@{}}
\toprule
{algorithm} & \multicolumn{2}{l}{{algo. time (s)}} & \multicolumn{2}{l}{{eval. time (s)}} & \multicolumn{2}{l}{{total time (s)}} \\
\midrule
& {mean} & {dev.} & {mean} & {dev.} & {mean} & {dev.} \\
\midrule
rls & 0.211039 & 0.004141 & 3.321902 & 0.010200 & 3.532941 & 0.010334 \\
 hc & 0.176032 & 0.003209 & 3.638269 & 0.012063 & 3.814301 & 0.012837 \\
 sa & 0.214841 & 0.004258 & 0.552141 & 0.041788 & 0.766982 & 0.042838 \\
 ea-1p1 & 0.312344 & 0.005768 & 0.952958 & 0.033976 & 1.265302 & 0.035527 \\
 ea-1p10 & 0.332388 & 0.002941 & 0.960034 & 0.054660 & 1.292422 & 0.054573 \\
 ea-10p1 & 0.390850 & 0.009494 & 1.503902 & 0.068603 & 1.894751 & 0.072589 \\
 ea-1c10 & 0.288653 & 0.002520 & 1.488057 & 0.042850 & 1.776710 & 0.042566 \\
 ga & 1.194618 & 0.003032 & 3.678091 & 0.034131 & 4.872709 & 0.034563 \\
 pbil & 1.307878 & 0.004923 & 1.960993 & 0.066584 & 3.268870 & 0.070404 \\
 umda & 1.241253 & 0.004591 & 0.809216 & 0.020771 & 2.050469 & 0.022456 \\
 \bottomrule
\end{tabular}

\end{center}

\begin{figure}[h]
\begin{center}
\includegraphics[width=0.6\linewidth]{labs}
\caption{labs}
\end{center}
\end{figure}

\begin{figure}[h]
\begin{center}
\includegraphics[width=0.6\linewidth]{labs+all}
\caption{labs}
\end{center}
\end{figure}

\newpage

\section{Function ep}

\begin{center}
\begin{tabular}{@{}l*{5}{>{{\nprounddigits{1}}}N{1}{1}}>{{\nprounddigits{0}}}N{2}{0}@{}}
\toprule
{algorithm} & \multicolumn{6}{l}{{function value}} \\
\midrule
& {min} & {$Q_1$} & {med.} & {$Q_3$} & {max} & {rk}\\
\midrule
rls & {\color{blue}} 4.555360e-32 & 1.017098e-30 & 2.354830e-30 & 3.780038e-30 & 8.782220e-30 & 2\\
hc & 1.316100e-31 & 9.264195e-31 & 2.800940e-30 & 5.405007e-30 & 1.428500e-29 & 3\\
sa & 1.616330e-31 & 3.087900e-30 & 5.393475e-30 & 1.083720e-29 & 1.722250e-29 & 6\\
ea-1p1 & 1.263020e-30 & 9.306558e-30 & 1.293505e-29 & 2.476340e-29 & 1.020120e-28 & 8\\
ea-1p10 & 1.985000e-30 & 9.738132e-30 & 1.395595e-29 & 4.550345e-29 & 7.443310e-29 & 9\\
ea-10p1 & 1.139560e-31 & 2.804805e-30 & 6.244055e-30 & 1.018357e-29 & 2.189540e-29 & 7\\
ea-1c10 & 2.146610e-31 & 3.176273e-30 & 4.444840e-30 & 1.295420e-29 & 2.009430e-29 & 4\\
ga & 2.068700e-31 & {\color{blue}} 7.900457e-31 & {\color{blue}} 1.796415e-30 & {\color{blue}} 2.989448e-30 & {\color{blue}} 7.205530e-30 & 1\\
pbil & 1.073810e-30 & 2.516122e-30 & 5.140770e-30 & 7.434193e-30 & 1.235050e-29 & 5\\
umda & 5.640520e-31 & 1.626207e-29 & 3.517360e-29 & 5.284237e-29 & 1.356960e-28 & 10\\
\bottomrule
\end{tabular}

\end{center}

\begin{center}
\begin{tabular}{@{}l*{6}{>{{\nprounddigits{2}}}N{1}{2}}@{}}
\toprule
{algorithm} & \multicolumn{2}{l}{{algo. time (s)}} & \multicolumn{2}{l}{{eval. time (s)}} & \multicolumn{2}{l}{{total time (s)}} \\
\midrule
& {mean} & {dev.} & {mean} & {dev.} & {mean} & {dev.} \\
\midrule
rls & 0.357059 & 0.018622 & 0.490487 & 0.026965 & 0.847546 & 0.045413 \\
 hc & 0.306882 & 0.004681 & 0.508342 & 0.005484 & 0.815224 & 0.004550 \\
 sa & 0.345611 & 0.003584 & 0.499773 & 0.008004 & 0.845383 & 0.006480 \\
 ea-1p1 & 0.448861 & 0.006947 & 0.496278 & 0.009700 & 0.945139 & 0.013493 \\
 ea-1p10 & 0.471261 & 0.002691 & 0.492505 & 0.005694 & 0.963765 & 0.006148 \\
 ea-10p1 & 0.548369 & 0.007911 & 0.507647 & 0.010642 & 1.056016 & 0.015381 \\
 ea-1c10 & 0.432435 & 0.013097 & 0.490365 & 0.004052 & 0.922800 & 0.014627 \\
 ga & 1.571415 & 0.017162 & 0.535071 & 0.007328 & 2.106486 & 0.017577 \\
 pbil & 1.707200 & 0.012335 & 0.546575 & 0.005869 & 2.253775 & 0.016606 \\
 umda & 1.557465 & 0.010265 & 0.463288 & 0.007214 & 2.020753 & 0.013113 \\
 \bottomrule
\end{tabular}

\end{center}

\begin{figure}[h]
\begin{center}
\includegraphics[width=0.6\linewidth]{ep}
\caption{ep}
\end{center}
\end{figure}

\begin{figure}[h]
\begin{center}
\includegraphics[width=0.6\linewidth]{ep+all}
\caption{ep}
\end{center}
\end{figure}

\newpage

\section{Function cancel}

\begin{center}
\begin{tabular}{@{}l*{5}{>{{\nprounddigits{3}}}N{1}{3}}>{{\nprounddigits{0}}}N{2}{0}@{}}
\toprule
{algorithm} & \multicolumn{6}{l}{{cache lookup ratio}} \\
\midrule
& {min} & {$Q_1$} & {med.} & {$Q_3$} & {max} & {rk}\\
\midrule
rls & 0.091033 & 0.093122 & 0.093737 & 0.094504 & 0.095363 & 8\\
hc & 0.018180 & 0.018217 & 0.018247 & 0.018270 & 0.018287 & 9\\
sa & 0.648457 & 0.688690 & 0.701984 & 0.719641 & 0.742680 & 2\\
ea-1p1 & 0.576617 & 0.655202 & 0.682235 & 0.749836 & 0.811747 & 3\\
ea-1p10 & 0.583617 & 0.616312 & 0.670083 & 0.729826 & 0.814540 & 4\\
ea-10p1 & 0.276263 & 0.414563 & 0.481345 & 0.541630 & 0.638113 & 7\\
ea-1c10 & 0.593090 & 0.600823 & 0.606295 & 0.618375 & 0.629737 & 6\\
ga & 0.006073 & 0.006330 & 0.006432 & 0.006550 & 0.007600 & 10\\
pbil & 0.548640 & 0.619876 & 0.633016 & 0.644915 & 0.662673 & 5\\
umda & {\color{blue}} 0.827807 & {\color{blue}} 0.838925 & {\color{blue}} 0.861062 & {\color{blue}} 0.874763 & {\color{blue}} 0.897113 & 1\\
\bottomrule
\end{tabular}

\end{center}

\begin{center}
\begin{tabular}{@{}l*{6}{>{{\nprounddigits{2}}}N{1}{2}}@{}}
\toprule
{algorithm} & \multicolumn{2}{l}{{algo. time (s)}} & \multicolumn{2}{l}{{eval. time (s)}} & \multicolumn{2}{l}{{total time (s)}} \\
\midrule
& {mean} & {dev.} & {mean} & {dev.} & {mean} & {dev.} \\
\midrule
rls & 0.330963 & 0.006498 & 0.500222 & 0.008046 & 0.831185 & 0.012224 \\
 hc & 0.296731 & 0.006013 & 0.500279 & 0.006577 & 0.797009 & 0.005809 \\
 sa & 0.346077 & 0.004472 & 0.483226 & 0.007642 & 0.829303 & 0.006889 \\
 ea-1p1 & 0.447966 & 0.005719 & 0.485508 & 0.008856 & 0.933474 & 0.009700 \\
 ea-1p10 & 0.469011 & 0.002827 & 0.479461 & 0.005221 & 0.948472 & 0.006757 \\
 ea-10p1 & 0.545165 & 0.007440 & 0.479162 & 0.010377 & 1.024327 & 0.016407 \\
 ea-1c10 & 0.427567 & 0.005019 & 0.477979 & 0.006354 & 0.905546 & 0.010716 \\
 ga & 1.560179 & 0.007800 & 0.466575 & 0.006932 & 2.026755 & 0.011222 \\
 pbil & 1.593433 & 0.004235 & 0.476608 & 0.005093 & 2.070041 & 0.005637 \\
 umda & 1.543424 & 0.023426 & 0.456518 & 0.007886 & 1.999942 & 0.026134 \\
 \bottomrule
\end{tabular}

\end{center}

\begin{figure}[h]
\begin{center}
\includegraphics[width=0.6\linewidth]{cancel}
\caption{cancel}
\end{center}
\end{figure}

\begin{figure}[h]
\begin{center}
\includegraphics[width=0.6\linewidth]{cancel+all}
\caption{cancel}
\end{center}
\end{figure}

\newpage

\section{Function trap}

\begin{center}
\begin{tabular}{@{}l*{5}{>{{\nprounddigits{0}}}N{3}{0}}>{{\nprounddigits{0}}}N{2}{0}@{}}
\toprule
{algorithm} & \multicolumn{6}{l}{{function value}} \\
\midrule
& {min} & {$Q_1$} & {med.} & {$Q_3$} & {max} & {rk}\\
\midrule
rls & {\color{blue}} 91.000000 & {\color{blue}} 91.000000 & {\color{blue}} 91.000000 & 91.000000 & {\color{blue}} 92.000000 & 2\\
hc & {\color{blue}} 91.000000 & {\color{blue}} 91.000000 & {\color{blue}} 91.000000 & {\color{blue}} 91.250000 & {\color{blue}} 92.000000 & 1\\
sa & 90.000000 & 90.000000 & 90.000000 & 90.000000 & 91.000000 & 4\\
ea-1p1 & 90.000000 & 90.000000 & 90.000000 & 90.000000 & 91.000000 & 4\\
ea-1p10 & 90.000000 & 90.000000 & 90.000000 & 90.000000 & 90.000000 & 7\\
ea-10p1 & 90.000000 & 90.000000 & 90.000000 & 90.000000 & {\color{blue}} 92.000000 & 3\\
ea-1c10 & 90.000000 & 90.000000 & 90.000000 & 90.000000 & 90.000000 & 7\\
ga & 90.000000 & 90.000000 & 90.000000 & 90.000000 & 91.000000 & 4\\
pbil & 90.000000 & 90.000000 & 90.000000 & 90.000000 & 90.000000 & 7\\
umda & 90.000000 & 90.000000 & 90.000000 & 90.000000 & 90.000000 & 7\\
\bottomrule
\end{tabular}

\end{center}

\begin{center}
\begin{tabular}{@{}l*{6}{>{{\nprounddigits{2}}}N{1}{2}}@{}}
\toprule
{algorithm} & \multicolumn{2}{l}{{algo. time (s)}} & \multicolumn{2}{l}{{eval. time (s)}} & \multicolumn{2}{l}{{total time (s)}} \\
\midrule
& {mean} & {dev.} & {mean} & {dev.} & {mean} & {dev.} \\
\midrule
rls & 0.204776 & 0.000804 & 0.303007 & 0.003127 & 0.507783 & 0.003501 \\
 hc & 0.178027 & 0.001091 & 0.301270 & 0.002586 & 0.479297 & 0.003063 \\
 sa & 0.212064 & 0.001740 & 0.303174 & 0.003402 & 0.515238 & 0.004688 \\
 ea-1p1 & 0.311157 & 0.004733 & 0.305414 & 0.004833 & 0.616571 & 0.005781 \\
 ea-1p10 & 0.332434 & 0.001653 & 0.308793 & 0.005251 & 0.641227 & 0.004633 \\
 ea-10p1 & 0.400115 & 0.009347 & 0.311545 & 0.004955 & 0.711660 & 0.010592 \\
 ea-1c10 & 0.288200 & 0.001597 & 0.306046 & 0.002688 & 0.594246 & 0.003362 \\
 ga & 1.198214 & 0.003508 & 0.313075 & 0.003940 & 1.511289 & 0.006403 \\
 pbil & 1.258403 & 0.002789 & 0.307409 & 0.002975 & 1.565812 & 0.003550 \\
 umda & 1.239880 & 0.003204 & 0.308025 & 0.002510 & 1.547905 & 0.004163 \\
 \bottomrule
\end{tabular}

\end{center}

\begin{figure}[h]
\begin{center}
\includegraphics[width=0.6\linewidth]{trap}
\caption{trap}
\end{center}
\end{figure}

\begin{figure}[h]
\begin{center}
\includegraphics[width=0.6\linewidth]{trap+all}
\caption{trap}
\end{center}
\end{figure}

\newpage

\section{Function hiff}

\begin{center}
\begin{tabular}{@{}l*{5}{>{{\nprounddigits{0}}}N{4}{0}}>{{\nprounddigits{0}}}N{2}{0}@{}}
\toprule
{algorithm} & \multicolumn{6}{l}{{function value}} \\
\midrule
& {min} & {$Q_1$} & {med.} & {$Q_3$} & {max} & {rk}\\
\midrule
rls & 400.000000 & 409.500000 & 418.000000 & 425.000000 & 446.000000 & 10\\
hc & 476.000000 & 498.000000 & 510.000000 & 516.000000 & 576.000000 & 6\\
sa & 332.000000 & 696.000000 & 704.000000 & 736.000000 & 768.000000 & 2\\
ea-1p1 & 432.000000 & 478.000000 & 488.000000 & 498.000000 & 560.000000 & 8\\
ea-1p10 & 432.000000 & 470.000000 & 480.000000 & 506.000000 & 552.000000 & 9\\
ea-10p1 & 560.000000 & 664.000000 & 684.000000 & 736.000000 & {\color{blue}} 1024.000000 & 3\\
ea-1c10 & 600.000000 & 639.000000 & 680.000000 & 716.000000 & 776.000000 & 4\\
ga & {\color{blue}} 640.000000 & {\color{blue}} 711.000000 & {\color{blue}} 746.000000 & {\color{blue}} 768.000000 & 800.000000 & 1\\
pbil & 476.000000 & 518.000000 & 541.000000 & 600.500000 & 680.000000 & 5\\
umda & 472.000000 & 493.000000 & 506.000000 & 521.000000 & 576.000000 & 7\\
\bottomrule
\end{tabular}

\end{center}

\begin{center}
\begin{tabular}{@{}l*{6}{>{{\nprounddigits{2}}}N{1}{2}}@{}}
\toprule
{algorithm} & \multicolumn{2}{l}{{algo. time (s)}} & \multicolumn{2}{l}{{eval. time (s)}} & \multicolumn{2}{l}{{total time (s)}} \\
\midrule
& {mean} & {dev.} & {mean} & {dev.} & {mean} & {dev.} \\
\midrule
rls & 0.210399 & 0.001186 & 0.616450 & 0.004073 & 0.826849 & 0.004410 \\
 hc & 0.175936 & 0.000941 & 0.625437 & 0.004173 & 0.801373 & 0.004635 \\
 sa & 0.203295 & 0.046143 & 0.681781 & 0.157139 & 0.885076 & 0.202668 \\
 ea-1p1 & 0.307637 & 0.002205 & 0.673590 & 0.011236 & 0.981227 & 0.011379 \\
 ea-1p10 & 0.330550 & 0.002157 & 0.682883 & 0.009781 & 1.013432 & 0.009841 \\
 ea-10p1 & 0.391243 & 0.009649 & 0.740833 & 0.015077 & 1.132076 & 0.019805 \\
 ea-1c10 & 0.288342 & 0.002175 & 0.698288 & 0.009683 & 0.986630 & 0.009918 \\
 ga & 1.369976 & 0.002767 & 0.774840 & 0.011747 & 2.144816 & 0.011889 \\
 pbil & 1.571334 & 0.006378 & 0.712164 & 0.013279 & 2.283498 & 0.015714 \\
 umda & 1.537203 & 0.007035 & 0.695470 & 0.015321 & 2.232673 & 0.016911 \\
 \bottomrule
\end{tabular}

\end{center}

\begin{figure}[h]
\begin{center}
\includegraphics[width=0.6\linewidth]{hiff}
\caption{hiff}
\end{center}
\end{figure}

\begin{figure}[h]
\begin{center}
\includegraphics[width=0.6\linewidth]{hiff+all}
\caption{hiff}
\end{center}
\end{figure}

\newpage

\section{Function plateau}

\begin{center}
\begin{tabular}{@{}l*{5}{>{{\nprounddigits{3}}}N{1}{3}}>{{\nprounddigits{0}}}N{2}{0}@{}}
\toprule
{algorithm} & \multicolumn{6}{l}{{cache lookup ratio}} \\
\midrule
& {min} & {$Q_1$} & {med.} & {$Q_3$} & {max} & {rk}\\
\midrule
rls & 0.142330 & 0.144638 & 0.145707 & 0.148333 & 0.151257 & 8\\
hc & 0.044363 & 0.045683 & 0.046173 & 0.046823 & 0.048090 & 10\\
sa & {\color{blue}} 0.918957 & {\color{blue}} 0.921750 & {\color{blue}} 0.926011 & {\color{blue}} 0.930418 & {\color{blue}} 0.936680 & 1\\
ea-1p1 & 0.596000 & 0.620041 & 0.634602 & 0.643318 & 0.745793 & 5\\
ea-1p10 & 0.598853 & 0.614446 & 0.631855 & 0.646502 & 0.673513 & 6\\
ea-10p1 & 0.136520 & 0.138119 & 0.138447 & 0.139375 & 0.140507 & 9\\
ea-1c10 & 0.803430 & 0.809833 & 0.812782 & 0.816239 & 0.821677 & 4\\
ga & 0.363757 & 0.370813 & 0.373845 & 0.375468 & 0.383733 & 7\\
pbil & 0.839947 & 0.846588 & 0.850620 & 0.850982 & 0.853237 & 2\\
umda & 0.798837 & 0.828993 & 0.835470 & 0.849448 & 0.874240 & 3\\
\bottomrule
\end{tabular}

\end{center}

\begin{center}
\begin{tabular}{@{}l*{6}{>{{\nprounddigits{2}}}N{1}{2}}@{}}
\toprule
{algorithm} & \multicolumn{2}{l}{{algo. time (s)}} & \multicolumn{2}{l}{{eval. time (s)}} & \multicolumn{2}{l}{{total time (s)}} \\
\midrule
& {mean} & {dev.} & {mean} & {dev.} & {mean} & {dev.} \\
\midrule
rls & 0.216297 & 0.001910 & 0.282076 & 0.004411 & 0.498373 & 0.005926 \\
 hc & 0.185967 & 0.000959 & 0.280214 & 0.005076 & 0.466181 & 0.005831 \\
 sa & 0.203975 & 0.037608 & 0.266826 & 0.048893 & 0.470801 & 0.086465 \\
 ea-1p1 & 0.321125 & 0.036826 & 0.291297 & 0.032944 & 0.612421 & 0.069270 \\
 ea-1p10 & 0.328856 & 0.062906 & 0.279514 & 0.053815 & 0.608370 & 0.116678 \\
 ea-10p1 & 0.455864 & 0.007635 & 0.295068 & 0.003381 & 0.750932 & 0.007331 \\
 ea-1c10 & 0.305302 & 0.001212 & 0.295367 & 0.003295 & 0.600669 & 0.003701 \\
 ga & 1.284496 & 0.002090 & 0.299434 & 0.002655 & 1.583930 & 0.002902 \\
 pbil & 1.298841 & 0.003670 & 0.298688 & 0.002717 & 1.597529 & 0.003999 \\
 umda & 1.277263 & 0.003963 & 0.300953 & 0.003579 & 1.578216 & 0.006501 \\
 \bottomrule
\end{tabular}

\end{center}

\begin{figure}[h]
\begin{center}
\includegraphics[width=0.6\linewidth]{plateau}
\caption{plateau}
\end{center}
\end{figure}

\begin{figure}[h]
\begin{center}
\includegraphics[width=0.6\linewidth]{plateau+all}
\caption{plateau}
\end{center}
\end{figure}

\newpage

\section{Function walsh2}

\begin{center}
\begin{tabular}{@{}l*{5}{>{{\nprounddigits{3}}}N{1}{3}}>{{\nprounddigits{0}}}N{2}{0}@{}}
\toprule
{algorithm} & \multicolumn{6}{l}{{cache lookup ratio}} \\
\midrule
& {min} & {$Q_1$} & {med.} & {$Q_3$} & {max} & {rk}\\
\midrule
rls & 0.095237 & 0.096362 & 0.096543 & 0.096783 & 0.098667 & 8\\
hc & 0.019197 & 0.019224 & 0.019232 & 0.019243 & 0.019263 & 10\\
sa & 0.844023 & 0.864458 & 0.872395 & 0.879029 & {\color{blue}} 0.902777 & 2\\
ea-1p1 & 0.815587 & 0.851956 & 0.861530 & 0.862881 & 0.864543 & 3\\
ea-1p10 & 0.818960 & 0.851491 & 0.856313 & 0.860855 & 0.862820 & 4\\
ea-10p1 & 0.601637 & 0.701998 & 0.712587 & 0.727063 & 0.734700 & 7\\
ea-1c10 & 0.709783 & 0.733367 & 0.756560 & 0.787418 & 0.838703 & 6\\
ga & 0.029883 & 0.038407 & 0.047405 & 0.062559 & 0.095947 & 9\\
pbil & 0.731740 & 0.772260 & 0.778620 & 0.785156 & 0.805240 & 5\\
umda & {\color{blue}} 0.882733 & {\color{blue}} 0.896031 & {\color{blue}} 0.896934 & {\color{blue}} 0.898959 & 0.901467 & 1\\
\bottomrule
\end{tabular}

\end{center}

\begin{center}
\begin{tabular}{@{}l*{6}{>{{\nprounddigits{2}}}N{1}{2}}@{}}
\toprule
{algorithm} & \multicolumn{2}{l}{{algo. time (s)}} & \multicolumn{2}{l}{{eval. time (s)}} & \multicolumn{2}{l}{{total time (s)}} \\
\midrule
& {mean} & {dev.} & {mean} & {dev.} & {mean} & {dev.} \\
\midrule
rls & 0.219870 & 0.003083 & 3.352350 & 0.021191 & 3.572220 & 0.021946 \\
 hc & 0.191065 & 0.002579 & 3.512989 & 0.049316 & 3.704054 & 0.051391 \\
 sa & 0.224234 & 0.004798 & 0.871900 & 0.050634 & 1.096134 & 0.053212 \\
 ea-1p1 & 0.321762 & 0.003806 & 1.009529 & 0.042848 & 1.331290 & 0.042133 \\
 ea-1p10 & 0.348853 & 0.006718 & 1.039426 & 0.040897 & 1.388278 & 0.043427 \\
 ea-10p1 & 0.411688 & 0.010754 & 1.611220 & 0.109281 & 2.022908 & 0.111897 \\
 ea-1c10 & 0.305465 & 0.004400 & 1.299005 & 0.111670 & 1.604470 & 0.112092 \\
 ga & 1.226177 & 0.011107 & 4.054885 & 0.076345 & 5.281062 & 0.077688 \\
 pbil & 1.280216 & 0.005132 & 1.379059 & 0.064675 & 2.659275 & 0.068150 \\
 umda & 1.232041 & 0.066526 & 0.842406 & 0.041122 & 2.074447 & 0.106580 \\
 \bottomrule
\end{tabular}

\end{center}

\begin{figure}[h]
\begin{center}
\includegraphics[width=0.6\linewidth]{walsh2}
\caption{walsh2}
\end{center}
\end{figure}

\begin{figure}[h]
\begin{center}
\includegraphics[width=0.6\linewidth]{walsh2+all}
\caption{walsh2}
\end{center}
\end{figure}

