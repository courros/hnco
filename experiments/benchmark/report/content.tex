\graphicspath{{../graphics/}}

\section{Rankings}

\begin{center}
\begin{tabular}{@{}l*{10}{r}@{}}
\toprule
algorithm & \multicolumn{10}{l}{{rank distribution}}\\
\midrule
& 1 & 2 & 3 & 4 & 5 & 6 & 7 & 8 & 9 & 10\\
\midrule
pbil & 10 & 0 & 1 & 0 & 3 & 0 & 2 & 0 & 1 & 2\\
hc & 7 & 3 & 0 & 2 & 2 & 1 & 1 & 0 & 0 & 3\\
umda & 7 & 2 & 0 & 0 & 2 & 1 & 2 & 1 & 2 & 2\\
ga & 5 & 4 & 2 & 0 & 3 & 2 & 0 & 0 & 1 & 2\\
sa & 5 & 3 & 2 & 3 & 1 & 0 & 1 & 1 & 2 & 1\\
ea-10p1 & 5 & 3 & 1 & 5 & 3 & 1 & 0 & 1 & 0 & 0\\
ea-1c10 & 5 & 2 & 2 & 4 & 3 & 1 & 1 & 1 & 0 & 0\\
rls & 4 & 6 & 0 & 2 & 2 & 2 & 0 & 1 & 1 & 1\\
ea-1p10 & 4 & 3 & 1 & 4 & 0 & 1 & 2 & 2 & 2 & 0\\
ea-1p1 & 4 & 3 & 1 & 3 & 1 & 1 & 1 & 2 & 2 & 1\\
\bottomrule
\end{tabular}

\end{center}

\newpage

\section{Function one-max}

\begin{center}
\begin{tabular}{@{}l*{5}{>{{\nprounddigits{0}}}N{3}{0}}>{{\nprounddigits{0}}}N{2}{0}N{1}{3}N{1}{3}@{}}
\toprule
{algorithm} & \multicolumn{6}{l}{{function value}} & \multicolumn{2}{l}{{time (s)}} \\
\midrule
& {min} & {$Q_1$} & {med.} & {$Q_3$} & {max} & {rk} & {mean} & {dev.} \\
\midrule
rls & {\color{blue}} 100.000000 & {\color{blue}} 100.000000 & {\color{blue}} 100.000000 & {\color{blue}} 100.000000 & {\color{blue}} 100.000000 & 1 & 0.000000 & 0.000000 \\
 hc & {\color{blue}} 100.000000 & {\color{blue}} 100.000000 & {\color{blue}} 100.000000 & {\color{blue}} 100.000000 & {\color{blue}} 100.000000 & 1 & 0.000000 & 0.000000 \\
 sa & {\color{blue}} 100.000000 & {\color{blue}} 100.000000 & {\color{blue}} 100.000000 & {\color{blue}} 100.000000 & {\color{blue}} 100.000000 & 1 & 0.002500 & 0.004443 \\
 ea-1p1 & {\color{blue}} 100.000000 & {\color{blue}} 100.000000 & {\color{blue}} 100.000000 & {\color{blue}} 100.000000 & {\color{blue}} 100.000000 & 1 & 0.000000 & 0.000000 \\
 ea-1p10 & {\color{blue}} 100.000000 & {\color{blue}} 100.000000 & {\color{blue}} 100.000000 & {\color{blue}} 100.000000 & {\color{blue}} 100.000000 & 1 & 0.000000 & 0.000000 \\
 ea-10p1 & {\color{blue}} 100.000000 & {\color{blue}} 100.000000 & {\color{blue}} 100.000000 & {\color{blue}} 100.000000 & {\color{blue}} 100.000000 & 1 & 0.005500 & 0.005104 \\
 ea-1c10 & {\color{blue}} 100.000000 & {\color{blue}} 100.000000 & {\color{blue}} 100.000000 & {\color{blue}} 100.000000 & {\color{blue}} 100.000000 & 1 & 0.000500 & 0.002236 \\
 ga & {\color{blue}} 100.000000 & {\color{blue}} 100.000000 & {\color{blue}} 100.000000 & {\color{blue}} 100.000000 & {\color{blue}} 100.000000 & 1 & 0.010500 & 0.002236 \\
 pbil & {\color{blue}} 100.000000 & {\color{blue}} 100.000000 & {\color{blue}} 100.000000 & {\color{blue}} 100.000000 & {\color{blue}} 100.000000 & 1 & 0.063000 & 0.006569 \\
 umda & {\color{blue}} 100.000000 & {\color{blue}} 100.000000 & {\color{blue}} 100.000000 & {\color{blue}} 100.000000 & {\color{blue}} 100.000000 & 1 & 0.001500 & 0.003663 \\
 \bottomrule
\end{tabular}

\end{center}

\begin{figure}[h]
\begin{center}
\includegraphics[width=0.6\linewidth]{one-max}
\caption{one-max}
\end{center}
\end{figure}

\begin{figure}[h]
\begin{center}
\includegraphics[width=0.6\linewidth]{one-max+all}
\caption{one-max}
\end{center}
\end{figure}

\newpage

\section{Function lin}

\begin{center}
\begin{tabular}{@{}l*{5}{>{{\nprounddigits{2}}}N{2}{2}}>{{\nprounddigits{0}}}N{2}{0}N{1}{3}N{1}{3}@{}}
\toprule
{algorithm} & \multicolumn{6}{l}{{function value}} & \multicolumn{2}{l}{{time (s)}} \\
\midrule
& {min} & {$Q_1$} & {med.} & {$Q_3$} & {max} & {rk} & {mean} & {dev.} \\
\midrule
rls & {\color{blue}} 45.032200 & {\color{blue}} 45.032200 & {\color{blue}} 45.032200 & {\color{blue}} 45.032200 & {\color{blue}} 45.032200 & 1 & 0.184000 & 0.005982 \\
 hc & {\color{blue}} 45.032200 & {\color{blue}} 45.032200 & {\color{blue}} 45.032200 & {\color{blue}} 45.032200 & {\color{blue}} 45.032200 & 1 & 0.152500 & 0.004443 \\
 sa & 26.925400 & {\color{blue}} 45.032200 & {\color{blue}} 45.032200 & {\color{blue}} 45.032200 & {\color{blue}} 45.032200 & 10 & 0.183000 & 0.004702 \\
 ea-1p1 & {\color{blue}} 45.032200 & {\color{blue}} 45.032200 & {\color{blue}} 45.032200 & {\color{blue}} 45.032200 & {\color{blue}} 45.032200 & 1 & 0.296500 & 0.011367 \\
 ea-1p10 & {\color{blue}} 45.032200 & {\color{blue}} 45.032200 & {\color{blue}} 45.032200 & {\color{blue}} 45.032200 & {\color{blue}} 45.032200 & 1 & 0.308500 & 0.007452 \\
 ea-10p1 & {\color{blue}} 45.032200 & {\color{blue}} 45.032200 & {\color{blue}} 45.032200 & {\color{blue}} 45.032200 & {\color{blue}} 45.032200 & 1 & 0.375000 & 0.008885 \\
 ea-1c10 & {\color{blue}} 45.032200 & {\color{blue}} 45.032200 & {\color{blue}} 45.032200 & {\color{blue}} 45.032200 & {\color{blue}} 45.032200 & 1 & 0.262500 & 0.006387 \\
 ga & {\color{blue}} 45.032200 & {\color{blue}} 45.032200 & {\color{blue}} 45.032200 & {\color{blue}} 45.032200 & {\color{blue}} 45.032200 & 1 & 1.227500 & 0.016504 \\
 pbil & {\color{blue}} 45.032200 & {\color{blue}} 45.032200 & {\color{blue}} 45.032200 & {\color{blue}} 45.032200 & {\color{blue}} 45.032200 & 1 & 1.238500 & 0.019541 \\
 umda & {\color{blue}} 45.032200 & {\color{blue}} 45.032200 & {\color{blue}} 45.032200 & {\color{blue}} 45.032200 & {\color{blue}} 45.032200 & 1 & 1.216000 & 0.016670 \\
 \bottomrule
\end{tabular}

\end{center}

\begin{figure}[h]
\begin{center}
\includegraphics[width=0.6\linewidth]{lin}
\caption{lin}
\end{center}
\end{figure}

\begin{figure}[h]
\begin{center}
\includegraphics[width=0.6\linewidth]{lin+all}
\caption{lin}
\end{center}
\end{figure}

\newpage

\section{Function leading-ones}

\begin{center}
\begin{tabular}{@{} l >{{\nprounddigits{0}}}n{6}{0}>{{\nprounddigits{2}}}n{6}{2}>{{\nprounddigits{1}}}n{6}{1}>{{\nprounddigits{2}}}n{6}{2}>{{\nprounddigits{0}}}n{6}{0} >{{\nprounddigits{1} \npunit{\%}}}n{3}{1} @{}}
\toprule
{Algorithm} & \multicolumn{5}{l}{{Number of evaluations}} & {Success} \\
\midrule
& {min} & {$Q_1$} & {med.} & {$Q_3$} & {max} & \\
\midrule
hc & 4398.000000 & 4626.000000 & 4751.000000 & 4900.750000 & 5398.000000 & 100\\
sa & 3281.000000 & 4714.250000 & 5248.000000 & 5620.000000 & 6111.000000 & 100\\
ea-1p1 & 4636.000000 & 5028.750000 & 5394.000000 & 6360.750000 & 6880.000000 & 100\\
ea-1p10 & 4184.000000 & 5443.750000 & 6023.500000 & 6356.500000 & 8378.000000 & 100\\
umda & 9426.000000 & 10577.000000 & 11585.000000 & 13052.750000 & 13862.000000 & 100\\
rls & 5118.000000 & 8015.750000 & 16811.000000 & 21852.500000 & 24621.000000 & 100\\
ea-10p1 & 32567.000000 & 34904.750000 & 39334.000000 & 42730.250000 & 45869.000000 & 100\\
pbil & 62521.000000 & 70371.000000 & 71776.500000 & 76118.500000 & 80962.000000 & 100\\
ga & 23262.000000 & 27195.000000 & 89687.500000 & 117401.000000 & 194344.000000 & 0\\
\bottomrule
\end{tabular}

\end{center}

\begin{figure}[h]
\begin{center}
\includegraphics[width=0.6\linewidth]{leading-ones}
\caption{leading-ones}
\end{center}
\end{figure}

\begin{figure}[h]
\begin{center}
\includegraphics[width=0.6\linewidth]{leading-ones+all}
\caption{leading-ones}
\end{center}
\end{figure}

\newpage

\section{Function ridge}

\begin{center}
\begin{tabular}{@{}l*{5}{>{{\nprounddigits{0}}}N{3}{0}}>{{\nprounddigits{0}}}N{2}{0}>{{\nprounddigits{2}}}N{1}{2}>{{\nprounddigits{2}}}N{1}{2}@{}}
\toprule
{algorithm} & \multicolumn{6}{l}{{function value}} & \multicolumn{2}{l}{{time (s)}} \\
\midrule
& {min} & {$Q_1$} & {med.} & {$Q_3$} & {max} & {rk} & {mean} & {dev.} \\
\midrule
rls & 104.000000 & 104.000000 & 105.000000 & 107.000000 & 110.000000 & 9 & 0.162500 & 0.020229 \\
 hc & {\color{blue}} 200.000000 & {\color{blue}} 200.000000 & {\color{blue}} 200.000000 & {\color{blue}} 200.000000 & {\color{blue}} 200.000000 & 1 & 0.004000 & 0.005026 \\
 sa & {\color{blue}} 200.000000 & {\color{blue}} 200.000000 & {\color{blue}} 200.000000 & {\color{blue}} 200.000000 & {\color{blue}} 200.000000 & 1 & 0.010000 & 0.003244 \\
 ea-1p1 & {\color{blue}} 200.000000 & {\color{blue}} 200.000000 & {\color{blue}} 200.000000 & {\color{blue}} 200.000000 & {\color{blue}} 200.000000 & 1 & 0.012000 & 0.004104 \\
 ea-1p10 & {\color{blue}} 200.000000 & {\color{blue}} 200.000000 & {\color{blue}} 200.000000 & {\color{blue}} 200.000000 & {\color{blue}} 200.000000 & 1 & 0.017500 & 0.004443 \\
 ea-10p1 & {\color{blue}} 200.000000 & {\color{blue}} 200.000000 & {\color{blue}} 200.000000 & {\color{blue}} 200.000000 & {\color{blue}} 200.000000 & 1 & 0.202000 & 0.026477 \\
 ea-1c10 & 118.000000 & 120.000000 & 124.500000 & 130.250000 & 144.000000 & 8 & 0.243500 & 0.006708 \\
 ga & 102.000000 & 102.000000 & 102.500000 & 103.000000 & 104.000000 & 10 & 1.218000 & 0.074240 \\
 pbil & 152.000000 & 153.000000 & 154.000000 & 155.000000 & 156.000000 & 7 & 1.236500 & 0.043682 \\
 umda & {\color{blue}} 200.000000 & {\color{blue}} 200.000000 & {\color{blue}} 200.000000 & {\color{blue}} 200.000000 & {\color{blue}} 200.000000 & 1 & 0.204500 & 0.019861 \\
 \bottomrule
\end{tabular}

\end{center}

\begin{figure}[h]
\begin{center}
\includegraphics[width=0.6\linewidth]{ridge}
\caption{ridge}
\end{center}
\end{figure}

\begin{figure}[h]
\begin{center}
\includegraphics[width=0.6\linewidth]{ridge+all}
\caption{ridge}
\end{center}
\end{figure}

\newpage

\section{Function jmp-5}

\begin{center}
\begin{tabular}{@{}l*{5}{>{{\nprounddigits{0}}}N{3}{0}}>{{\nprounddigits{0}}}N{2}{0}>{{\nprounddigits{2}}}N{1}{2}>{{\nprounddigits{2}}}N{1}{2}@{}}
\toprule
{algorithm} & \multicolumn{6}{l}{{function value}} & \multicolumn{2}{l}{{time (s)}} \\
\midrule
& {min} & {$Q_1$} & {med.} & {$Q_3$} & {max} & {rk} & {mean} & {dev.} \\
\midrule
rls & 95.000000 & 95.000000 & 95.000000 & 95.000000 & 95.000000 & 4 & 0.146000 & 0.013917 \\
 hc & 95.000000 & 95.000000 & 95.000000 & 95.000000 & 95.000000 & 4 & 0.125000 & 0.010513 \\
 sa & 95.000000 & 95.000000 & 95.000000 & 95.000000 & 95.000000 & 4 & 0.155500 & 0.010990 \\
 ea-1p1 & 95.000000 & 95.000000 & 95.000000 & 95.000000 & 95.000000 & 4 & 0.262000 & 0.023974 \\
 ea-1p10 & 95.000000 & 95.000000 & 95.000000 & 95.000000 & 95.000000 & 4 & 0.294000 & 0.037191 \\
 ea-10p1 & 95.000000 & 95.000000 & 95.000000 & 95.000000 & 95.000000 & 4 & 0.375500 & 0.085068 \\
 ea-1c10 & 95.000000 & 95.000000 & 95.000000 & 95.000000 & 95.000000 & 4 & 0.298000 & 0.058273 \\
 ga & {\color{blue}} 100.000000 & {\color{blue}} 100.000000 & {\color{blue}} 100.000000 & {\color{blue}} 100.000000 & {\color{blue}} 100.000000 & 1 & 0.393000 & 0.325222 \\
 pbil & {\color{blue}} 100.000000 & {\color{blue}} 100.000000 & {\color{blue}} 100.000000 & {\color{blue}} 100.000000 & {\color{blue}} 100.000000 & 1 & 0.078000 & 0.013611 \\
 umda & {\color{blue}} 100.000000 & {\color{blue}} 100.000000 & {\color{blue}} 100.000000 & {\color{blue}} 100.000000 & {\color{blue}} 100.000000 & 1 & 0.198000 & 0.221659 \\
 \bottomrule
\end{tabular}

\end{center}

\begin{figure}[h]
\begin{center}
\includegraphics[width=0.6\linewidth]{jmp-5}
\caption{jmp-5}
\end{center}
\end{figure}

\begin{figure}[h]
\begin{center}
\includegraphics[width=0.6\linewidth]{jmp-5+all}
\caption{jmp-5}
\end{center}
\end{figure}

\newpage

\section{Function jmp-10}

\begin{center}
\begin{tabular}{@{}l*{5}{>{{\nprounddigits{0}}}N{3}{0}}>{{\nprounddigits{0}}}N{2}{0}N{1}{3}N{1}{3}@{}}
\toprule
{algorithm} & \multicolumn{6}{l}{{function value}} & \multicolumn{2}{l}{{time (s)}} \\
\midrule
& {min} & {$Q_1$} & {med.} & {$Q_3$} & {max} & {rk} & {mean} & {dev.} \\
\midrule
rls & {\color{blue}} 90.000000 & {\color{blue}} 90.000000 & {\color{blue}} 90.000000 & 90.000000 & 90.000000 & 2 & 0.155000 & 0.014327 \\
 hc & {\color{blue}} 90.000000 & {\color{blue}} 90.000000 & {\color{blue}} 90.000000 & 90.000000 & 90.000000 & 2 & 0.121000 & 0.003078 \\
 sa & {\color{blue}} 90.000000 & {\color{blue}} 90.000000 & {\color{blue}} 90.000000 & 90.000000 & 90.000000 & 2 & 0.171000 & 0.033230 \\
 ea-1p1 & {\color{blue}} 90.000000 & {\color{blue}} 90.000000 & {\color{blue}} 90.000000 & 90.000000 & 90.000000 & 2 & 0.283500 & 0.058424 \\
 ea-1p10 & {\color{blue}} 90.000000 & {\color{blue}} 90.000000 & {\color{blue}} 90.000000 & 90.000000 & 90.000000 & 2 & 0.276500 & 0.014244 \\
 ea-10p1 & {\color{blue}} 90.000000 & {\color{blue}} 90.000000 & {\color{blue}} 90.000000 & 90.000000 & 90.000000 & 2 & 0.352000 & 0.041244 \\
 ea-1c10 & {\color{blue}} 90.000000 & {\color{blue}} 90.000000 & {\color{blue}} 90.000000 & 90.000000 & 90.000000 & 2 & 0.242500 & 0.035670 \\
 ga & {\color{blue}} 90.000000 & {\color{blue}} 90.000000 & {\color{blue}} 90.000000 & 90.000000 & 90.000000 & 2 & 1.191500 & 0.046597 \\
 pbil & {\color{blue}} 90.000000 & {\color{blue}} 90.000000 & {\color{blue}} 90.000000 & {\color{blue}} 100.000000 & {\color{blue}} 100.000000 & 1 & 0.938500 & 0.501716 \\
 umda & {\color{blue}} 90.000000 & {\color{blue}} 90.000000 & {\color{blue}} 90.000000 & 90.000000 & 90.000000 & 2 & 1.273500 & 0.066354 \\
 \bottomrule
\end{tabular}

\end{center}

\begin{figure}[h]
\begin{center}
\includegraphics[width=0.6\linewidth]{jmp-10}
\caption{jmp-10}
\end{center}
\end{figure}

\begin{figure}[h]
\begin{center}
\includegraphics[width=0.6\linewidth]{jmp-10+all}
\caption{jmp-10}
\end{center}
\end{figure}

\newpage

\section{Function djmp-5}

\begin{center}
\begin{tabular}{@{}l*{5}{>{{\nprounddigits{0}}}N{3}{0}}>{{\nprounddigits{0}}}N{2}{0}>{{\nprounddigits{2}}}N{1}{2}>{{\nprounddigits{2}}}N{1}{2}@{}}
\toprule
{algorithm} & \multicolumn{6}{l}{{function value}} & \multicolumn{2}{l}{{time (s)}} \\
\midrule
& {min} & {$Q_1$} & {med.} & {$Q_3$} & {max} & {rk} & {mean} & {dev.} \\
\midrule
rls & 100.000000 & 100.000000 & 100.000000 & 100.000000 & 100.000000 & 4 & 0.165500 & 0.023725 \\
 hc & 100.000000 & 100.000000 & 100.000000 & 100.000000 & 100.000000 & 4 & 0.164500 & 0.038041 \\
 sa & 100.000000 & 100.000000 & 100.000000 & 100.000000 & 100.000000 & 4 & 0.188000 & 0.030018 \\
 ea-1p1 & 100.000000 & 100.000000 & 100.000000 & 100.000000 & 100.000000 & 4 & 0.353500 & 0.076315 \\
 ea-1p10 & 100.000000 & 100.000000 & 100.000000 & 100.000000 & 100.000000 & 4 & 0.276000 & 0.006806 \\
 ea-10p1 & 100.000000 & 100.000000 & 100.000000 & 100.000000 & 100.000000 & 4 & 0.342500 & 0.010195 \\
 ea-1c10 & 100.000000 & 100.000000 & 100.000000 & 100.000000 & 100.000000 & 4 & 0.263500 & 0.058515 \\
 ga & {\color{blue}} 105.000000 & {\color{blue}} 105.000000 & {\color{blue}} 105.000000 & {\color{blue}} 105.000000 & {\color{blue}} 105.000000 & 1 & 0.541500 & 0.396355 \\
 pbil & {\color{blue}} 105.000000 & {\color{blue}} 105.000000 & {\color{blue}} 105.000000 & {\color{blue}} 105.000000 & {\color{blue}} 105.000000 & 1 & 0.079000 & 0.019440 \\
 umda & {\color{blue}} 105.000000 & {\color{blue}} 105.000000 & {\color{blue}} 105.000000 & {\color{blue}} 105.000000 & {\color{blue}} 105.000000 & 1 & 0.186000 & 0.158758 \\
 \bottomrule
\end{tabular}

\end{center}

\begin{figure}[h]
\begin{center}
\includegraphics[width=0.6\linewidth]{djmp-5}
\caption{djmp-5}
\end{center}
\end{figure}

\begin{figure}[h]
\begin{center}
\includegraphics[width=0.6\linewidth]{djmp-5+all}
\caption{djmp-5}
\end{center}
\end{figure}

\newpage

\section{Function djmp-10}

\begin{center}
\begin{tabular}{@{}l*{5}{>{{\nprounddigits{0}}}N{3}{0}}>{{\nprounddigits{0}}}N{2}{0}>{{\nprounddigits{2}}}N{1}{2}>{{\nprounddigits{2}}}N{1}{2}@{}}
\toprule
{algorithm} & \multicolumn{6}{l}{{function value}} & \multicolumn{2}{l}{{time (s)}} \\
\midrule
& {min} & {$Q_1$} & {med.} & {$Q_3$} & {max} & {rk} & {mean} & {dev.} \\
\midrule
rls & {\color{blue}} 100.000000 & {\color{blue}} 100.000000 & 100.000000 & 100.000000 & 100.000000 & 2 & 0.159000 & 0.019708 \\
 hc & {\color{blue}} 100.000000 & {\color{blue}} 100.000000 & 100.000000 & 100.000000 & 100.000000 & 2 & 0.120500 & 0.006048 \\
 sa & {\color{blue}} 100.000000 & {\color{blue}} 100.000000 & 100.000000 & 100.000000 & 100.000000 & 2 & 0.151000 & 0.004472 \\
 ea-1p1 & {\color{blue}} 100.000000 & {\color{blue}} 100.000000 & 100.000000 & 100.000000 & 100.000000 & 2 & 0.251500 & 0.009333 \\
 ea-1p10 & {\color{blue}} 100.000000 & {\color{blue}} 100.000000 & 100.000000 & 100.000000 & 100.000000 & 2 & 0.279500 & 0.002236 \\
 ea-10p1 & {\color{blue}} 100.000000 & {\color{blue}} 100.000000 & 100.000000 & 100.000000 & 100.000000 & 2 & 0.343000 & 0.006569 \\
 ea-1c10 & {\color{blue}} 100.000000 & {\color{blue}} 100.000000 & 100.000000 & 100.000000 & 100.000000 & 2 & 0.277000 & 0.027928 \\
 ga & {\color{blue}} 100.000000 & {\color{blue}} 100.000000 & 100.000000 & 100.000000 & 100.000000 & 2 & 1.403500 & 0.090045 \\
 pbil & {\color{blue}} 100.000000 & {\color{blue}} 100.000000 & {\color{blue}} 110.000000 & {\color{blue}} 110.000000 & {\color{blue}} 110.000000 & 1 & 0.596500 & 0.587773 \\
 umda & {\color{blue}} 100.000000 & {\color{blue}} 100.000000 & 100.000000 & 100.000000 & 100.000000 & 2 & 1.230000 & 0.016543 \\
 \bottomrule
\end{tabular}

\end{center}

\begin{figure}[h]
\begin{center}
\includegraphics[width=0.6\linewidth]{djmp-10}
\caption{djmp-10}
\end{center}
\end{figure}

\begin{figure}[h]
\begin{center}
\includegraphics[width=0.6\linewidth]{djmp-10+all}
\caption{djmp-10}
\end{center}
\end{figure}

\newpage

\section{Function fp-5}

\begin{center}
\begin{tabular}{@{}l*{5}{>{{\nprounddigits{0}}}N{3}{0}}>{{\nprounddigits{0}}}N{2}{0}N{1}{3}N{1}{3}@{}}
\toprule
{algorithm} & \multicolumn{6}{l}{{function value}} & \multicolumn{2}{l}{{time (s)}} \\
\midrule
& {min} & {$Q_1$} & {med.} & {$Q_3$} & {max} & {rk} & {mean} & {dev.} \\
\midrule
rls & {\color{blue}} 194.000000 & {\color{blue}} 194.000000 & {\color{blue}} 194.000000 & {\color{blue}} 194.000000 & {\color{blue}} 194.000000 & 1 & 0.004500 & 0.009445 \\
 hc & 100.000000 & 100.000000 & 100.000000 & {\color{blue}} 194.000000 & {\color{blue}} 194.000000 & 10 & 0.090000 & 0.037417 \\
 sa & 4.000000 & 100.000000 & {\color{blue}} 194.000000 & {\color{blue}} 194.000000 & {\color{blue}} 194.000000 & 8 & 0.057500 & 0.065363 \\
 ea-1p1 & {\color{blue}} 194.000000 & {\color{blue}} 194.000000 & {\color{blue}} 194.000000 & {\color{blue}} 194.000000 & {\color{blue}} 194.000000 & 1 & 0.001000 & 0.003078 \\
 ea-1p10 & 100.000000 & {\color{blue}} 194.000000 & {\color{blue}} 194.000000 & {\color{blue}} 194.000000 & {\color{blue}} 194.000000 & 7 & 0.016000 & 0.066917 \\
 ea-10p1 & {\color{blue}} 194.000000 & {\color{blue}} 194.000000 & {\color{blue}} 194.000000 & {\color{blue}} 194.000000 & {\color{blue}} 194.000000 & 1 & 0.043500 & 0.011367 \\
 ea-1c10 & {\color{blue}} 194.000000 & {\color{blue}} 194.000000 & {\color{blue}} 194.000000 & {\color{blue}} 194.000000 & {\color{blue}} 194.000000 & 1 & 0.004500 & 0.005104 \\
 ga & 187.000000 & 189.000000 & 190.000000 & 191.250000 & {\color{blue}} 194.000000 & 9 & 1.080500 & 0.273909 \\
 pbil & {\color{blue}} 194.000000 & {\color{blue}} 194.000000 & {\color{blue}} 194.000000 & {\color{blue}} 194.000000 & {\color{blue}} 194.000000 & 1 & 0.368000 & 0.036792 \\
 umda & {\color{blue}} 194.000000 & {\color{blue}} 194.000000 & {\color{blue}} 194.000000 & {\color{blue}} 194.000000 & {\color{blue}} 194.000000 & 1 & 0.044500 & 0.006863 \\
 \bottomrule
\end{tabular}

\end{center}

\begin{figure}[h]
\begin{center}
\includegraphics[width=0.6\linewidth]{fp-5}
\caption{fp-5}
\end{center}
\end{figure}

\begin{figure}[h]
\begin{center}
\includegraphics[width=0.6\linewidth]{fp-5+all}
\caption{fp-5}
\end{center}
\end{figure}

\newpage

\section{Function fp-10}

\begin{center}
\begin{tabular}{@{}l*{5}{>{{\nprounddigits{0}}}N{3}{0}}>{{\nprounddigits{0}}}N{2}{0}N{1}{3}N{1}{3}@{}}
\toprule
{algorithm} & \multicolumn{6}{l}{{function value}} & \multicolumn{2}{l}{{time (s)}} \\
\midrule
& {min} & {$Q_1$} & {med.} & {$Q_3$} & {max} & {rk} & {mean} & {dev.} \\
\midrule
rls & 188.000000 & {\color{blue}} 189.000000 & {\color{blue}} 189.000000 & {\color{blue}} 189.000000 & {\color{blue}} 189.000000 & 2 & 0.055000 & 0.047406 \\
 hc & 100.000000 & 100.000000 & 100.000000 & 100.000000 & 100.000000 & 10 & 0.120000 & 0.003244 \\
 sa & 100.000000 & 100.000000 & 100.000000 & 122.250000 & {\color{blue}} 189.000000 & 7 & 0.117000 & 0.067207 \\
 ea-1p1 & 100.000000 & 100.000000 & 100.000000 & 100.000000 & {\color{blue}} 189.000000 & 9 & 0.257000 & 0.060880 \\
 ea-1p10 & 100.000000 & 100.000000 & 100.000000 & 122.250000 & {\color{blue}} 189.000000 & 7 & 0.211000 & 0.122814 \\
 ea-10p1 & 100.000000 & 166.750000 & {\color{blue}} 189.000000 & {\color{blue}} 189.000000 & {\color{blue}} 189.000000 & 4 & 0.118500 & 0.135035 \\
 ea-1c10 & 100.000000 & {\color{blue}} 189.000000 & {\color{blue}} 189.000000 & {\color{blue}} 189.000000 & {\color{blue}} 189.000000 & 3 & 0.091000 & 0.087714 \\
 ga & 182.000000 & 184.000000 & 185.000000 & 187.000000 & {\color{blue}} 189.000000 & 5 & 1.168000 & 0.115376 \\
 pbil & {\color{blue}} 189.000000 & {\color{blue}} 189.000000 & {\color{blue}} 189.000000 & {\color{blue}} 189.000000 & {\color{blue}} 189.000000 & 1 & 0.307000 & 0.022734 \\
 umda & 100.000000 & 100.000000 & 159.000000 & {\color{blue}} 189.000000 & {\color{blue}} 189.000000 & 6 & 0.625000 & 0.595240 \\
 \bottomrule
\end{tabular}

\end{center}

\begin{figure}[h]
\begin{center}
\includegraphics[width=0.6\linewidth]{fp-10}
\caption{fp-10}
\end{center}
\end{figure}

\begin{figure}[h]
\begin{center}
\includegraphics[width=0.6\linewidth]{fp-10+all}
\caption{fp-10}
\end{center}
\end{figure}

\newpage

\section{Function nk}

\begin{center}
\begin{tabular}{@{}l*{5}{>{{\nprounddigits{2}}}N{1}{2}}>{{\nprounddigits{0}}}N{2}{0}>{{\nprounddigits{2}}}N{1}{2}>{{\nprounddigits{2}}}N{1}{2}@{}}
\toprule
{algorithm} & \multicolumn{6}{l}{{function value}} & \multicolumn{2}{l}{{time (s)}} \\
\midrule
& {min} & {$Q_1$} & {med.} & {$Q_3$} & {max} & {rk} & {mean} & {dev.} \\
\midrule
rls & 0.966108 & 0.976354 & 0.985223 & 1.010020 & 1.033870 & 7 & 0.627500 & 0.011180 \\
 hc & 0.959273 & 0.974589 & 0.998966 & 1.020720 & 1.050620 & 5 & 0.579000 & 0.005525 \\
 sa & {\color{blue}} 1.011040 & {\color{blue}} 1.034975 & {\color{blue}} 1.044625 & {\color{blue}} 1.063945 & {\color{blue}} 1.102760 & 1 & 0.710500 & 0.075775 \\
 ea-1p1 & 0.865587 & 0.922936 & 0.947898 & 0.980645 & 1.044850 & 8 & 0.887000 & 0.045665 \\
 ea-1p10 & 0.818509 & 0.903203 & 0.942312 & 0.979686 & 1.100350 & 9 & 0.784500 & 0.030689 \\
 ea-10p1 & 0.948038 & 0.978267 & 0.999527 & 1.011367 & 1.064590 & 4 & 0.855000 & 0.027242 \\
 ea-1c10 & 0.961991 & 0.982793 & 1.016820 & 1.036067 & 1.084590 & 3 & 0.669500 & 0.011459 \\
 ga & 0.957025 & 1.012088 & 1.024390 & 1.046277 & 1.085990 & 2 & 2.091500 & 0.151528 \\
 pbil & 0.919099 & 0.980689 & 0.998383 & 1.014750 & 1.047210 & 6 & 1.701500 & 0.027198 \\
 umda & 0.841614 & 0.907093 & 0.936000 & 0.959514 & 1.009240 & 10 & 1.640500 & 0.027237 \\
 \bottomrule
\end{tabular}

\end{center}

\begin{figure}[h]
\begin{center}
\includegraphics[width=0.6\linewidth]{nk}
\caption{nk}
\end{center}
\end{figure}

\begin{figure}[h]
\begin{center}
\includegraphics[width=0.6\linewidth]{nk+all}
\caption{nk}
\end{center}
\end{figure}

\newpage

\section{Function max-sat}

\begin{center}
\begin{tabular}{@{}l*{5}{>{{\nprounddigits{0}}}N{3}{0}}>{{\nprounddigits{0}}}N{2}{0}>{{\nprounddigits{2}}}N{1}{2}>{{\nprounddigits{2}}}N{1}{2}@{}}
\toprule
{algorithm} & \multicolumn{6}{l}{{function value}} & \multicolumn{2}{l}{{time (s)}} \\
\midrule
& {min} & {$Q_1$} & {med.} & {$Q_3$} & {max} & {rk} & {mean} & {dev.} \\
\midrule
rls & {\color{blue}} 970.000000 & 971.000000 & {\color{blue}} 972.000000 & {\color{blue}} 972.000000 & {\color{blue}} 972.000000 & 2 & 3.506000 & 0.052656 \\
 hc & 964.000000 & 966.000000 & 967.000000 & 968.000000 & 971.000000 & 10 & 3.410000 & 0.266399 \\
 sa & 969.000000 & {\color{blue}} 972.000000 & {\color{blue}} 972.000000 & {\color{blue}} 972.000000 & {\color{blue}} 972.000000 & 1 & 2.990500 & 0.075007 \\
 ea-1p1 & 963.000000 & 964.750000 & 967.500000 & 970.250000 & {\color{blue}} 972.000000 & 8 & 3.502000 & 0.349881 \\
 ea-1p10 & 961.000000 & 967.000000 & 968.000000 & 969.250000 & {\color{blue}} 972.000000 & 6 & 3.505000 & 0.315336 \\
 ea-10p1 & 960.000000 & 967.750000 & 968.500000 & 971.250000 & {\color{blue}} 972.000000 & 5 & 4.372500 & 0.367321 \\
 ea-1c10 & 964.000000 & 968.750000 & 970.000000 & {\color{blue}} 972.000000 & {\color{blue}} 972.000000 & 3 & 2.964000 & 0.142585 \\
 ga & 964.000000 & 967.750000 & 968.500000 & {\color{blue}} 972.000000 & {\color{blue}} 972.000000 & 4 & 5.350000 & 0.131669 \\
 pbil & 965.000000 & 967.000000 & 967.000000 & 967.000000 & 969.000000 & 9 & 4.618000 & 0.346526 \\
 umda & 953.000000 & 965.000000 & 967.500000 & 970.250000 & {\color{blue}} 972.000000 & 7 & 4.456000 & 0.363700 \\
 \bottomrule
\end{tabular}

\end{center}

\begin{figure}[h]
\begin{center}
\includegraphics[width=0.6\linewidth]{max-sat}
\caption{max-sat}
\end{center}
\end{figure}

\begin{figure}[h]
\begin{center}
\includegraphics[width=0.6\linewidth]{max-sat+all}
\caption{max-sat}
\end{center}
\end{figure}

\newpage

\section{Function labs}

\begin{center}
\begin{tabular}{@{}l*{5}{>{{\nprounddigits{2}}}N{1}{2}}>{{\nprounddigits{0}}}N{2}{0}>{{\nprounddigits{2}}}N{1}{2}>{{\nprounddigits{2}}}N{1}{2}@{}}
\toprule
{algorithm} & \multicolumn{6}{l}{{function value}} & \multicolumn{2}{l}{{time (s)}} \\
\midrule
& {min} & {$Q_1$} & {med.} & {$Q_3$} & {max} & {rk} & {mean} & {dev.} \\
\midrule
rls & 4.201680 & 4.440500 & 4.504500 & 4.660830 & 4.990020 & 6 & 3.543000 & 0.151314 \\
 hc & 4.504500 & 4.599827 & 4.690695 & 4.995020 & 5.446620 & 4 & 3.506000 & 0.211247 \\
 sa & 4.347830 & 4.633933 & 4.780395 & 4.921782 & 5.285410 & 3 & 3.623500 & 0.270930 \\
 ea-1p1 & 3.840250 & 4.025760 & 4.152835 & 4.340635 & 4.970180 & 8 & 3.546000 & 0.311658 \\
 ea-1p10 & 3.607500 & 3.996810 & 4.187600 & 4.317947 & 4.520800 & 7 & 3.615000 & 0.236788 \\
 ea-10p1 & 4.332760 & 4.545497 & 4.646855 & 4.752900 & 5.263160 & 5 & 3.696500 & 0.306204 \\
 ea-1c10 & 4.553730 & 4.712650 & 4.873370 & {\color{blue}} 5.144097 & {\color{blue}} 5.668930 & 2 & 3.631000 & 0.347274 \\
 ga & {\color{blue}} 4.621070 & {\color{blue}} 4.854370 & {\color{blue}} 4.931045 & 5.086610 & 5.592840 & 1 & 4.507500 & 0.408513 \\
 pbil & 3.136760 & 3.852080 & 4.038770 & 4.191120 & 4.347830 & 9 & 4.761500 & 0.329789 \\
 umda & 3.364740 & 3.725780 & 3.975565 & 4.281778 & 4.816960 & 10 & 4.610500 & 0.332035 \\
 \bottomrule
\end{tabular}

\end{center}

\begin{figure}[h]
\begin{center}
\includegraphics[width=0.6\linewidth]{labs}
\caption{labs}
\end{center}
\end{figure}

\begin{figure}[h]
\begin{center}
\includegraphics[width=0.6\linewidth]{labs+all}
\caption{labs}
\end{center}
\end{figure}

\newpage

\section{Function ep}

\begin{center}
\begin{tabular}{@{}l*{5}{>{{\nprounddigits{1}}}N{1}{1}}>{{\nprounddigits{0}}}N{2}{0}N{1}{3}N{1}{3}@{}}
\toprule
{algorithm} & \multicolumn{6}{l}{{function value}} & \multicolumn{2}{l}{{time (s)}} \\
\midrule
& {min} & {$Q_1$} & {med.} & {$Q_3$} & {max} & {rk} & {mean} & {dev.} \\
\midrule
rls & {\color{blue}} 1.483610e-32 & {\color{blue}} 9.867337e-31 & {\color{blue}} 1.522215e-30 & {\color{blue}} 2.816145e-30 & {\color{blue}} 9.255170e-30 & 1 & 0.206000 & 0.005026 \\
 hc & 1.310510e-31 & 2.059752e-30 & 3.658035e-30 & 8.355073e-30 & 1.232640e-29 & 3 & 0.160000 & 0.000000 \\
 sa & 4.159670e-31 & 3.584670e-30 & 4.536765e-30 & 9.259082e-30 & 2.888500e-25 & 5 & 0.183500 & 0.011367 \\
 ea-1p1 & 2.050370e-31 & 6.891072e-30 & 1.362705e-29 & 1.720378e-29 & 4.998470e-29 & 8 & 0.291500 & 0.005871 \\
 ea-1p10 & 2.565570e-31 & 7.308730e-30 & 2.992075e-29 & 4.162682e-29 & 6.836570e-29 & 10 & 0.323500 & 0.004894 \\
 ea-10p1 & 7.963300e-31 & 3.962105e-30 & 7.263680e-30 & 1.103995e-29 & 2.099830e-29 & 6 & 0.404000 & 0.009947 \\
 ea-1c10 & 3.992100e-31 & 2.066820e-30 & 7.295400e-30 & 1.243635e-29 & 2.210560e-29 & 7 & 0.351000 & 0.006407 \\
 ga & 2.720780e-31 & 1.533487e-30 & 2.429595e-30 & 3.414335e-30 & 1.382220e-29 & 2 & 1.492500 & 0.115343 \\
 pbil & 4.574500e-31 & 1.997437e-30 & 4.377215e-30 & 8.594050e-30 & 2.123110e-29 & 4 & 1.457000 & 0.037571 \\
 umda & 1.416310e-30 & 1.296881e-29 & 2.644800e-29 & 5.522843e-29 & 1.368440e-28 & 9 & 1.248000 & 0.018525 \\
 \bottomrule
\end{tabular}

\end{center}

\begin{figure}[h]
\begin{center}
\includegraphics[width=0.6\linewidth]{ep}
\caption{ep}
\end{center}
\end{figure}

\begin{figure}[h]
\begin{center}
\includegraphics[width=0.6\linewidth]{ep+all}
\caption{ep}
\end{center}
\end{figure}

\newpage

\section{Function cancel}

\begin{center}
\begin{tabular}{@{}l*{5}{>{{\nprounddigits{2}}}N{1}{2}}>{{\nprounddigits{0}}}N{2}{0}N{1}{3}N{1}{3}@{}}
\toprule
{algorithm} & \multicolumn{6}{l}{{function value}} & \multicolumn{2}{l}{{time (s)}} \\
\midrule
& {min} & {$Q_1$} & {med.} & {$Q_3$} & {max} & {rk} & {mean} & {dev.} \\
\midrule
rls & 0.620000 & 1.372500 & 1.525000 & 1.692500 & 2.290000 & 8 & 0.182000 & 0.005231 \\
 hc & 1.240000 & 2.497500 & 3.610000 & 4.487500 & 7.490000 & 10 & 0.146000 & 0.005982 \\
 sa & 0.440000 & 1.247500 & 2.475000 & 3.212500 & 3.980000 & 9 & 0.184000 & 0.005026 \\
 ea-1p1 & 0.060000 & 0.125000 & 0.235000 & 0.692500 & 1.500000 & 2 & 0.303500 & 0.008751 \\
 ea-1p10 & 0.050000 & 0.307500 & 0.685000 & 0.762500 & 1.790000 & 4 & 0.311000 & 0.007881 \\
 ea-10p1 & 0.060000 & 0.200000 & 0.690000 & 0.807500 & 1.640000 & 5 & 0.380000 & 0.009177 \\
 ea-1c10 & 0.060000 & 0.410000 & 0.745000 & 1.362500 & 2.660000 & 6 & 0.262000 & 0.006959 \\
 ga & {\color{blue}} 0.030000 & 0.077500 & 0.540000 & 0.760000 & 1.670000 & 3 & 1.223500 & 0.017554 \\
 pbil & 0.040000 & {\color{blue}} 0.060000 & {\color{blue}} 0.070000 & {\color{blue}} 0.095000 & {\color{blue}} 0.320000 & 1 & 1.279000 & 0.016512 \\
 umda & 0.160000 & 1.137500 & 1.460000 & 1.907500 & 2.430000 & 7 & 1.215000 & 0.016384 \\
 \bottomrule
\end{tabular}

\end{center}

\begin{figure}[h]
\begin{center}
\includegraphics[width=0.6\linewidth]{cancel}
\caption{cancel}
\end{center}
\end{figure}

\begin{figure}[h]
\begin{center}
\includegraphics[width=0.6\linewidth]{cancel+all}
\caption{cancel}
\end{center}
\end{figure}

\newpage

\section{Function trap}

\begin{center}
\begin{tabular}{@{}l*{5}{>{{\nprounddigits{0}}}N{3}{0}}>{{\nprounddigits{0}}}N{2}{0}>{{\nprounddigits{2}}}N{1}{2}>{{\nprounddigits{2}}}N{1}{2}@{}}
\toprule
{algorithm} & \multicolumn{6}{l}{{function value}} & \multicolumn{2}{l}{{time (s)}} \\
\midrule
& {min} & {$Q_1$} & {med.} & {$Q_3$} & {max} & {rk} & {mean} & {dev.} \\
\midrule
rls & 90.000000 & {\color{blue}} 91.000000 & {\color{blue}} 91.000000 & 91.000000 & {\color{blue}} 92.000000 & 2 & 0.230500 & 0.008256 \\
 hc & {\color{blue}} 91.000000 & {\color{blue}} 91.000000 & {\color{blue}} 91.000000 & {\color{blue}} 92.000000 & {\color{blue}} 92.000000 & 1 & 0.198500 & 0.003663 \\
 sa & 90.000000 & 90.000000 & 90.000000 & 90.000000 & 91.000000 & 3 & 0.231000 & 0.014473 \\
 ea-1p1 & 90.000000 & 90.000000 & 90.000000 & 90.000000 & 90.000000 & 7 & 0.333000 & 0.004702 \\
 ea-1p10 & 90.000000 & 90.000000 & 90.000000 & 90.000000 & 91.000000 & 3 & 0.351000 & 0.003078 \\
 ea-10p1 & 90.000000 & 90.000000 & 90.000000 & 90.000000 & 91.000000 & 3 & 0.393500 & 0.006708 \\
 ea-1c10 & 90.000000 & 90.000000 & 90.000000 & 90.000000 & 90.000000 & 7 & 0.268500 & 0.003663 \\
 ga & 90.000000 & 90.000000 & 90.000000 & 90.000000 & 91.000000 & 3 & 1.208000 & 0.017045 \\
 pbil & 90.000000 & 90.000000 & 90.000000 & 90.000000 & 90.000000 & 7 & 1.457000 & 0.141945 \\
 umda & 90.000000 & 90.000000 & 90.000000 & 90.000000 & 90.000000 & 7 & 1.318000 & 0.070904 \\
 \bottomrule
\end{tabular}

\end{center}

\begin{figure}[h]
\begin{center}
\includegraphics[width=0.6\linewidth]{trap}
\caption{trap}
\end{center}
\end{figure}

\begin{figure}[h]
\begin{center}
\includegraphics[width=0.6\linewidth]{trap+all}
\caption{trap}
\end{center}
\end{figure}

\newpage

\section{Function hiff}

\begin{center}
\begin{tabular}{@{}l*{5}{>{{\nprounddigits{0}}}N{3}{0}}>{{\nprounddigits{0}}}N{2}{0}N{1}{3}N{1}{3}@{}}
\toprule
{algorithm} & \multicolumn{6}{l}{{function value}} & \multicolumn{2}{l}{{time (s)}} \\
\midrule
& {min} & {$Q_1$} & {med.} & {$Q_3$} & {max} & {rk} & {mean} & {dev.} \\
\midrule
rls & 402.000000 & 416.000000 & 422.000000 & 428.500000 & 448.000000 & 10 & 0.532500 & 0.014824 \\
 hc & 476.000000 & 484.000000 & 504.000000 & 516.000000 & 560.000000 & 6 & 0.498000 & 0.011517 \\
 sa & 640.000000 & 704.000000 & 704.000000 & 768.000000 & 832.000000 & 3 & 0.628000 & 0.017045 \\
 ea-1p1 & 440.000000 & 478.000000 & 492.000000 & 514.000000 & 552.000000 & 8 & 0.689000 & 0.019708 \\
 ea-1p10 & 416.000000 & 462.000000 & 476.000000 & 496.000000 & 584.000000 & 9 & 0.682500 & 0.023141 \\
 ea-10p1 & 632.000000 & 684.000000 & 736.000000 & {\color{blue}} 800.000000 & {\color{blue}} 896.000000 & 2 & 0.820000 & 0.018918 \\
 ea-1c10 & 612.000000 & 643.000000 & 664.000000 & 682.000000 & 760.000000 & 4 & 0.667000 & 0.014546 \\
 ga & {\color{blue}} 704.000000 & {\color{blue}} 717.000000 & {\color{blue}} 768.000000 & 772.000000 & 776.000000 & 1 & 1.839000 & 0.025935 \\
 pbil & 474.000000 & 508.000000 & 544.000000 & 560.500000 & 642.000000 & 5 & 1.933000 & 0.022266 \\
 umda & 456.000000 & 472.000000 & 502.000000 & 522.000000 & 560.000000 & 7 & 1.869500 & 0.017614 \\
 \bottomrule
\end{tabular}

\end{center}

\begin{figure}[h]
\begin{center}
\includegraphics[width=0.6\linewidth]{hiff}
\caption{hiff}
\end{center}
\end{figure}

\begin{figure}[h]
\begin{center}
\includegraphics[width=0.6\linewidth]{hiff+all}
\caption{hiff}
\end{center}
\end{figure}

\newpage

\section{Function plateau}

\begin{center}
\begin{tabular}{@{}l*{5}{>{{\nprounddigits{0}}}N{3}{0}}>{{\nprounddigits{0}}}N{2}{0}N{1}{3}N{1}{3}@{}}
\toprule
{algorithm} & \multicolumn{6}{l}{{function value}} & \multicolumn{2}{l}{{time (s)}} \\
\midrule
& {min} & {$Q_1$} & {med.} & {$Q_3$} & {max} & {rk} & {mean} & {dev.} \\
\midrule
rls & {\color{blue}} 101.000000 & {\color{blue}} 101.000000 & {\color{blue}} 101.000000 & 101.000000 & 101.000000 & 5 & 0.157000 & 0.008013 \\
 hc & {\color{blue}} 101.000000 & {\color{blue}} 101.000000 & {\color{blue}} 101.000000 & 101.250000 & {\color{blue}} 102.000000 & 2 & 0.131000 & 0.018325 \\
 sa & {\color{blue}} 101.000000 & {\color{blue}} 101.000000 & {\color{blue}} 101.000000 & {\color{blue}} 102.000000 & {\color{blue}} 102.000000 & 1 & 0.145500 & 0.045477 \\
 ea-1p1 & {\color{blue}} 101.000000 & {\color{blue}} 101.000000 & {\color{blue}} 101.000000 & 101.000000 & {\color{blue}} 102.000000 & 4 & 0.260000 & 0.054483 \\
 ea-1p10 & {\color{blue}} 101.000000 & {\color{blue}} 101.000000 & {\color{blue}} 101.000000 & 101.250000 & {\color{blue}} 102.000000 & 2 & 0.281500 & 0.023005 \\
 ea-10p1 & {\color{blue}} 101.000000 & {\color{blue}} 101.000000 & {\color{blue}} 101.000000 & 101.000000 & 101.000000 & 5 & 0.349000 & 0.010208 \\
 ea-1c10 & {\color{blue}} 101.000000 & {\color{blue}} 101.000000 & {\color{blue}} 101.000000 & 101.000000 & 101.000000 & 5 & 0.243000 & 0.004702 \\
 ga & {\color{blue}} 101.000000 & {\color{blue}} 101.000000 & {\color{blue}} 101.000000 & 101.000000 & 101.000000 & 5 & 1.187500 & 0.011642 \\
 pbil & {\color{blue}} 101.000000 & {\color{blue}} 101.000000 & {\color{blue}} 101.000000 & 101.000000 & 101.000000 & 5 & 1.215000 & 0.013179 \\
 umda & {\color{blue}} 101.000000 & {\color{blue}} 101.000000 & {\color{blue}} 101.000000 & 101.000000 & 101.000000 & 5 & 1.207000 & 0.028116 \\
 \bottomrule
\end{tabular}

\end{center}

\begin{figure}[h]
\begin{center}
\includegraphics[width=0.6\linewidth]{plateau}
\caption{plateau}
\end{center}
\end{figure}

\begin{figure}[h]
\begin{center}
\includegraphics[width=0.6\linewidth]{plateau+all}
\caption{plateau}
\end{center}
\end{figure}

\newpage

\section{Function walsh2}

\begin{center}
\begin{tabular}{@{}l*{5}{>{{\nprounddigits{2}}}N{3}{2}}>{{\nprounddigits{0}}}N{2}{0}>{{\nprounddigits{2}}}N{1}{2}>{{\nprounddigits{2}}}N{1}{2}@{}}
\toprule
{algorithm} & \multicolumn{6}{l}{{function value}} & \multicolumn{2}{l}{{time (s)}} \\
\midrule
& {min} & {$Q_1$} & {med.} & {$Q_3$} & {max} & {rk} & {mean} & {dev.} \\
\midrule
rls & 694.416000 & 700.638250 & 705.998500 & 712.735000 & 720.037000 & 3 & 3.346500 & 0.208106 \\
 hc & {\color{blue}} 700.775000 & 709.842250 & 714.578000 & {\color{blue}} 720.393000 & {\color{blue}} 721.219000 & 2 & 3.032500 & 0.118627 \\
 sa & 698.679000 & {\color{blue}} 713.693000 & {\color{blue}} 716.966500 & 720.240000 & {\color{blue}} 721.219000 & 1 & 3.534000 & 0.217023 \\
 ea-1p1 & 611.343000 & 651.462250 & 673.431000 & 688.512250 & 705.434000 & 7 & 3.623500 & 0.200349 \\
 ea-1p10 & 596.306000 & 649.747000 & 669.555500 & 688.042250 & 716.574000 & 8 & 3.586500 & 0.172086 \\
 ea-10p1 & 653.240000 & 686.419250 & 696.828000 & 703.146500 & 715.232000 & 6 & 4.125000 & 0.303428 \\
 ea-1c10 & 658.548000 & 688.132000 & 703.118500 & 714.077750 & 720.240000 & 4 & 3.353000 & 0.119212 \\
 ga & 682.783000 & 698.232000 & 702.218500 & 713.693000 & {\color{blue}} 721.219000 & 5 & 5.147000 & 0.149282 \\
 pbil & 623.150000 & 659.151250 & 664.803500 & 686.258750 & 710.755000 & 10 & 4.594500 & 0.175093 \\
 umda & 604.837000 & 646.325000 & 667.647000 & 682.117250 & 699.032000 & 9 & 4.271500 & 0.163233 \\
 \bottomrule
\end{tabular}

\end{center}

\begin{figure}[h]
\begin{center}
\includegraphics[width=0.6\linewidth]{walsh2}
\caption{walsh2}
\end{center}
\end{figure}

\begin{figure}[h]
\begin{center}
\includegraphics[width=0.6\linewidth]{walsh2+all}
\caption{walsh2}
\end{center}
\end{figure}

